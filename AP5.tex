\documentclass[11pt]{article}

    \usepackage[breakable]{tcolorbox}
    \usepackage{parskip} % Stop auto-indenting (to mimic markdown behaviour)
    

    % Basic figure setup, for now with no caption control since it's done
    % automatically by Pandoc (which extracts ![](path) syntax from Markdown).
    \usepackage{graphicx}
    % Maintain compatibility with old templates. Remove in nbconvert 6.0
    \let\Oldincludegraphics\includegraphics
    % Ensure that by default, figures have no caption (until we provide a
    % proper Figure object with a Caption API and a way to capture that
    % in the conversion process - todo).
    \usepackage{caption}
    \DeclareCaptionFormat{nocaption}{}
    \captionsetup{format=nocaption,aboveskip=0pt,belowskip=0pt}

    \usepackage{float}
    \floatplacement{figure}{H} % forces figures to be placed at the correct location
    \usepackage{xcolor} % Allow colors to be defined
    \usepackage{enumerate} % Needed for markdown enumerations to work
    \usepackage{geometry} % Used to adjust the document margins
    \usepackage{amsmath} % Equations
    \usepackage{amssymb} % Equations
    \usepackage{textcomp} % defines textquotesingle
    % Hack from http://tex.stackexchange.com/a/47451/13684:
    \AtBeginDocument{%
        \def\PYZsq{\textquotesingle}% Upright quotes in Pygmentized code
    }
    \usepackage{upquote} % Upright quotes for verbatim code
    \usepackage{eurosym} % defines \euro

    \usepackage{iftex}
    \ifPDFTeX
        \usepackage[T1]{fontenc}
        \IfFileExists{alphabeta.sty}{
              \usepackage{alphabeta}
          }{
              \usepackage[mathletters]{ucs}
              \usepackage[utf8x]{inputenc}
          }
    \else
        \usepackage{fontspec}
        \usepackage{unicode-math}
    \fi

    \usepackage{fancyvrb} % verbatim replacement that allows latex
    \usepackage{grffile} % extends the file name processing of package graphics
                         % to support a larger range
    \makeatletter % fix for old versions of grffile with XeLaTeX
    \@ifpackagelater{grffile}{2019/11/01}
    {
      % Do nothing on new versions
    }
    {
      \def\Gread@@xetex#1{%
        \IfFileExists{"\Gin@base".bb}%
        {\Gread@eps{\Gin@base.bb}}%
        {\Gread@@xetex@aux#1}%
      }
    }
    \makeatother
    \usepackage[Export]{adjustbox} % Used to constrain images to a maximum size
    \adjustboxset{max size={0.9\linewidth}{0.9\paperheight}}

    % The hyperref package gives us a pdf with properly built
    % internal navigation ('pdf bookmarks' for the table of contents,
    % internal cross-reference links, web links for URLs, etc.)
    \usepackage{hyperref}
    % The default LaTeX title has an obnoxious amount of whitespace. By default,
    % titling removes some of it. It also provides customization options.
    \usepackage{titling}
    \usepackage{longtable} % longtable support required by pandoc >1.10
    \usepackage{booktabs}  % table support for pandoc > 1.12.2
    \usepackage{array}     % table support for pandoc >= 2.11.3
    \usepackage{calc}      % table minipage width calculation for pandoc >= 2.11.1
    \usepackage[inline]{enumitem} % IRkernel/repr support (it uses the enumerate* environment)
    \usepackage[normalem]{ulem} % ulem is needed to support strikethroughs (\sout)
                                % normalem makes italics be italics, not underlines
    \usepackage{mathrsfs}
    

    
    % Colors for the hyperref package
    \definecolor{urlcolor}{rgb}{0,.145,.698}
    \definecolor{linkcolor}{rgb}{.71,0.21,0.01}
    \definecolor{citecolor}{rgb}{.12,.54,.11}

    % ANSI colors
    \definecolor{ansi-black}{HTML}{3E424D}
    \definecolor{ansi-black-intense}{HTML}{282C36}
    \definecolor{ansi-red}{HTML}{E75C58}
    \definecolor{ansi-red-intense}{HTML}{B22B31}
    \definecolor{ansi-green}{HTML}{00A250}
    \definecolor{ansi-green-intense}{HTML}{007427}
    \definecolor{ansi-yellow}{HTML}{DDB62B}
    \definecolor{ansi-yellow-intense}{HTML}{B27D12}
    \definecolor{ansi-blue}{HTML}{208FFB}
    \definecolor{ansi-blue-intense}{HTML}{0065CA}
    \definecolor{ansi-magenta}{HTML}{D160C4}
    \definecolor{ansi-magenta-intense}{HTML}{A03196}
    \definecolor{ansi-cyan}{HTML}{60C6C8}
    \definecolor{ansi-cyan-intense}{HTML}{258F8F}
    \definecolor{ansi-white}{HTML}{C5C1B4}
    \definecolor{ansi-white-intense}{HTML}{A1A6B2}
    \definecolor{ansi-default-inverse-fg}{HTML}{FFFFFF}
    \definecolor{ansi-default-inverse-bg}{HTML}{000000}

    % common color for the border for error outputs.
    \definecolor{outerrorbackground}{HTML}{FFDFDF}

    % commands and environments needed by pandoc snippets
    % extracted from the output of `pandoc -s`
    \providecommand{\tightlist}{%
      \setlength{\itemsep}{0pt}\setlength{\parskip}{0pt}}
    \DefineVerbatimEnvironment{Highlighting}{Verbatim}{commandchars=\\\{\}}
    % Add ',fontsize=\small' for more characters per line
    \newenvironment{Shaded}{}{}
    \newcommand{\KeywordTok}[1]{\textcolor[rgb]{0.00,0.44,0.13}{\textbf{{#1}}}}
    \newcommand{\DataTypeTok}[1]{\textcolor[rgb]{0.56,0.13,0.00}{{#1}}}
    \newcommand{\DecValTok}[1]{\textcolor[rgb]{0.25,0.63,0.44}{{#1}}}
    \newcommand{\BaseNTok}[1]{\textcolor[rgb]{0.25,0.63,0.44}{{#1}}}
    \newcommand{\FloatTok}[1]{\textcolor[rgb]{0.25,0.63,0.44}{{#1}}}
    \newcommand{\CharTok}[1]{\textcolor[rgb]{0.25,0.44,0.63}{{#1}}}
    \newcommand{\StringTok}[1]{\textcolor[rgb]{0.25,0.44,0.63}{{#1}}}
    \newcommand{\CommentTok}[1]{\textcolor[rgb]{0.38,0.63,0.69}{\textit{{#1}}}}
    \newcommand{\OtherTok}[1]{\textcolor[rgb]{0.00,0.44,0.13}{{#1}}}
    \newcommand{\AlertTok}[1]{\textcolor[rgb]{1.00,0.00,0.00}{\textbf{{#1}}}}
    \newcommand{\FunctionTok}[1]{\textcolor[rgb]{0.02,0.16,0.49}{{#1}}}
    \newcommand{\RegionMarkerTok}[1]{{#1}}
    \newcommand{\ErrorTok}[1]{\textcolor[rgb]{1.00,0.00,0.00}{\textbf{{#1}}}}
    \newcommand{\NormalTok}[1]{{#1}}

    % Additional commands for more recent versions of Pandoc
    \newcommand{\ConstantTok}[1]{\textcolor[rgb]{0.53,0.00,0.00}{{#1}}}
    \newcommand{\SpecialCharTok}[1]{\textcolor[rgb]{0.25,0.44,0.63}{{#1}}}
    \newcommand{\VerbatimStringTok}[1]{\textcolor[rgb]{0.25,0.44,0.63}{{#1}}}
    \newcommand{\SpecialStringTok}[1]{\textcolor[rgb]{0.73,0.40,0.53}{{#1}}}
    \newcommand{\ImportTok}[1]{{#1}}
    \newcommand{\DocumentationTok}[1]{\textcolor[rgb]{0.73,0.13,0.13}{\textit{{#1}}}}
    \newcommand{\AnnotationTok}[1]{\textcolor[rgb]{0.38,0.63,0.69}{\textbf{\textit{{#1}}}}}
    \newcommand{\CommentVarTok}[1]{\textcolor[rgb]{0.38,0.63,0.69}{\textbf{\textit{{#1}}}}}
    \newcommand{\VariableTok}[1]{\textcolor[rgb]{0.10,0.09,0.49}{{#1}}}
    \newcommand{\ControlFlowTok}[1]{\textcolor[rgb]{0.00,0.44,0.13}{\textbf{{#1}}}}
    \newcommand{\OperatorTok}[1]{\textcolor[rgb]{0.40,0.40,0.40}{{#1}}}
    \newcommand{\BuiltInTok}[1]{{#1}}
    \newcommand{\ExtensionTok}[1]{{#1}}
    \newcommand{\PreprocessorTok}[1]{\textcolor[rgb]{0.74,0.48,0.00}{{#1}}}
    \newcommand{\AttributeTok}[1]{\textcolor[rgb]{0.49,0.56,0.16}{{#1}}}
    \newcommand{\InformationTok}[1]{\textcolor[rgb]{0.38,0.63,0.69}{\textbf{\textit{{#1}}}}}
    \newcommand{\WarningTok}[1]{\textcolor[rgb]{0.38,0.63,0.69}{\textbf{\textit{{#1}}}}}


    % Define a nice break command that doesn't care if a line doesn't already
    % exist.
    \def\br{\hspace*{\fill} \\* }
    % Math Jax compatibility definitions
    \def\gt{>}
    \def\lt{<}
    \let\Oldtex\TeX
    \let\Oldlatex\LaTeX
    \renewcommand{\TeX}{\textrm{\Oldtex}}
    \renewcommand{\LaTeX}{\textrm{\Oldlatex}}
    % Document parameters
    % Document title
    \title{AP5}
    
    
    
    
    
    
    
% Pygments definitions
\makeatletter
\def\PY@reset{\let\PY@it=\relax \let\PY@bf=\relax%
    \let\PY@ul=\relax \let\PY@tc=\relax%
    \let\PY@bc=\relax \let\PY@ff=\relax}
\def\PY@tok#1{\csname PY@tok@#1\endcsname}
\def\PY@toks#1+{\ifx\relax#1\empty\else%
    \PY@tok{#1}\expandafter\PY@toks\fi}
\def\PY@do#1{\PY@bc{\PY@tc{\PY@ul{%
    \PY@it{\PY@bf{\PY@ff{#1}}}}}}}
\def\PY#1#2{\PY@reset\PY@toks#1+\relax+\PY@do{#2}}

\@namedef{PY@tok@w}{\def\PY@tc##1{\textcolor[rgb]{0.73,0.73,0.73}{##1}}}
\@namedef{PY@tok@c}{\let\PY@it=\textit\def\PY@tc##1{\textcolor[rgb]{0.24,0.48,0.48}{##1}}}
\@namedef{PY@tok@cp}{\def\PY@tc##1{\textcolor[rgb]{0.61,0.40,0.00}{##1}}}
\@namedef{PY@tok@k}{\let\PY@bf=\textbf\def\PY@tc##1{\textcolor[rgb]{0.00,0.50,0.00}{##1}}}
\@namedef{PY@tok@kp}{\def\PY@tc##1{\textcolor[rgb]{0.00,0.50,0.00}{##1}}}
\@namedef{PY@tok@kt}{\def\PY@tc##1{\textcolor[rgb]{0.69,0.00,0.25}{##1}}}
\@namedef{PY@tok@o}{\def\PY@tc##1{\textcolor[rgb]{0.40,0.40,0.40}{##1}}}
\@namedef{PY@tok@ow}{\let\PY@bf=\textbf\def\PY@tc##1{\textcolor[rgb]{0.67,0.13,1.00}{##1}}}
\@namedef{PY@tok@nb}{\def\PY@tc##1{\textcolor[rgb]{0.00,0.50,0.00}{##1}}}
\@namedef{PY@tok@nf}{\def\PY@tc##1{\textcolor[rgb]{0.00,0.00,1.00}{##1}}}
\@namedef{PY@tok@nc}{\let\PY@bf=\textbf\def\PY@tc##1{\textcolor[rgb]{0.00,0.00,1.00}{##1}}}
\@namedef{PY@tok@nn}{\let\PY@bf=\textbf\def\PY@tc##1{\textcolor[rgb]{0.00,0.00,1.00}{##1}}}
\@namedef{PY@tok@ne}{\let\PY@bf=\textbf\def\PY@tc##1{\textcolor[rgb]{0.80,0.25,0.22}{##1}}}
\@namedef{PY@tok@nv}{\def\PY@tc##1{\textcolor[rgb]{0.10,0.09,0.49}{##1}}}
\@namedef{PY@tok@no}{\def\PY@tc##1{\textcolor[rgb]{0.53,0.00,0.00}{##1}}}
\@namedef{PY@tok@nl}{\def\PY@tc##1{\textcolor[rgb]{0.46,0.46,0.00}{##1}}}
\@namedef{PY@tok@ni}{\let\PY@bf=\textbf\def\PY@tc##1{\textcolor[rgb]{0.44,0.44,0.44}{##1}}}
\@namedef{PY@tok@na}{\def\PY@tc##1{\textcolor[rgb]{0.41,0.47,0.13}{##1}}}
\@namedef{PY@tok@nt}{\let\PY@bf=\textbf\def\PY@tc##1{\textcolor[rgb]{0.00,0.50,0.00}{##1}}}
\@namedef{PY@tok@nd}{\def\PY@tc##1{\textcolor[rgb]{0.67,0.13,1.00}{##1}}}
\@namedef{PY@tok@s}{\def\PY@tc##1{\textcolor[rgb]{0.73,0.13,0.13}{##1}}}
\@namedef{PY@tok@sd}{\let\PY@it=\textit\def\PY@tc##1{\textcolor[rgb]{0.73,0.13,0.13}{##1}}}
\@namedef{PY@tok@si}{\let\PY@bf=\textbf\def\PY@tc##1{\textcolor[rgb]{0.64,0.35,0.47}{##1}}}
\@namedef{PY@tok@se}{\let\PY@bf=\textbf\def\PY@tc##1{\textcolor[rgb]{0.67,0.36,0.12}{##1}}}
\@namedef{PY@tok@sr}{\def\PY@tc##1{\textcolor[rgb]{0.64,0.35,0.47}{##1}}}
\@namedef{PY@tok@ss}{\def\PY@tc##1{\textcolor[rgb]{0.10,0.09,0.49}{##1}}}
\@namedef{PY@tok@sx}{\def\PY@tc##1{\textcolor[rgb]{0.00,0.50,0.00}{##1}}}
\@namedef{PY@tok@m}{\def\PY@tc##1{\textcolor[rgb]{0.40,0.40,0.40}{##1}}}
\@namedef{PY@tok@gh}{\let\PY@bf=\textbf\def\PY@tc##1{\textcolor[rgb]{0.00,0.00,0.50}{##1}}}
\@namedef{PY@tok@gu}{\let\PY@bf=\textbf\def\PY@tc##1{\textcolor[rgb]{0.50,0.00,0.50}{##1}}}
\@namedef{PY@tok@gd}{\def\PY@tc##1{\textcolor[rgb]{0.63,0.00,0.00}{##1}}}
\@namedef{PY@tok@gi}{\def\PY@tc##1{\textcolor[rgb]{0.00,0.52,0.00}{##1}}}
\@namedef{PY@tok@gr}{\def\PY@tc##1{\textcolor[rgb]{0.89,0.00,0.00}{##1}}}
\@namedef{PY@tok@ge}{\let\PY@it=\textit}
\@namedef{PY@tok@gs}{\let\PY@bf=\textbf}
\@namedef{PY@tok@ges}{\let\PY@bf=\textbf\let\PY@it=\textit}
\@namedef{PY@tok@gp}{\let\PY@bf=\textbf\def\PY@tc##1{\textcolor[rgb]{0.00,0.00,0.50}{##1}}}
\@namedef{PY@tok@go}{\def\PY@tc##1{\textcolor[rgb]{0.44,0.44,0.44}{##1}}}
\@namedef{PY@tok@gt}{\def\PY@tc##1{\textcolor[rgb]{0.00,0.27,0.87}{##1}}}
\@namedef{PY@tok@err}{\def\PY@bc##1{{\setlength{\fboxsep}{\string -\fboxrule}\fcolorbox[rgb]{1.00,0.00,0.00}{1,1,1}{\strut ##1}}}}
\@namedef{PY@tok@kc}{\let\PY@bf=\textbf\def\PY@tc##1{\textcolor[rgb]{0.00,0.50,0.00}{##1}}}
\@namedef{PY@tok@kd}{\let\PY@bf=\textbf\def\PY@tc##1{\textcolor[rgb]{0.00,0.50,0.00}{##1}}}
\@namedef{PY@tok@kn}{\let\PY@bf=\textbf\def\PY@tc##1{\textcolor[rgb]{0.00,0.50,0.00}{##1}}}
\@namedef{PY@tok@kr}{\let\PY@bf=\textbf\def\PY@tc##1{\textcolor[rgb]{0.00,0.50,0.00}{##1}}}
\@namedef{PY@tok@bp}{\def\PY@tc##1{\textcolor[rgb]{0.00,0.50,0.00}{##1}}}
\@namedef{PY@tok@fm}{\def\PY@tc##1{\textcolor[rgb]{0.00,0.00,1.00}{##1}}}
\@namedef{PY@tok@vc}{\def\PY@tc##1{\textcolor[rgb]{0.10,0.09,0.49}{##1}}}
\@namedef{PY@tok@vg}{\def\PY@tc##1{\textcolor[rgb]{0.10,0.09,0.49}{##1}}}
\@namedef{PY@tok@vi}{\def\PY@tc##1{\textcolor[rgb]{0.10,0.09,0.49}{##1}}}
\@namedef{PY@tok@vm}{\def\PY@tc##1{\textcolor[rgb]{0.10,0.09,0.49}{##1}}}
\@namedef{PY@tok@sa}{\def\PY@tc##1{\textcolor[rgb]{0.73,0.13,0.13}{##1}}}
\@namedef{PY@tok@sb}{\def\PY@tc##1{\textcolor[rgb]{0.73,0.13,0.13}{##1}}}
\@namedef{PY@tok@sc}{\def\PY@tc##1{\textcolor[rgb]{0.73,0.13,0.13}{##1}}}
\@namedef{PY@tok@dl}{\def\PY@tc##1{\textcolor[rgb]{0.73,0.13,0.13}{##1}}}
\@namedef{PY@tok@s2}{\def\PY@tc##1{\textcolor[rgb]{0.73,0.13,0.13}{##1}}}
\@namedef{PY@tok@sh}{\def\PY@tc##1{\textcolor[rgb]{0.73,0.13,0.13}{##1}}}
\@namedef{PY@tok@s1}{\def\PY@tc##1{\textcolor[rgb]{0.73,0.13,0.13}{##1}}}
\@namedef{PY@tok@mb}{\def\PY@tc##1{\textcolor[rgb]{0.40,0.40,0.40}{##1}}}
\@namedef{PY@tok@mf}{\def\PY@tc##1{\textcolor[rgb]{0.40,0.40,0.40}{##1}}}
\@namedef{PY@tok@mh}{\def\PY@tc##1{\textcolor[rgb]{0.40,0.40,0.40}{##1}}}
\@namedef{PY@tok@mi}{\def\PY@tc##1{\textcolor[rgb]{0.40,0.40,0.40}{##1}}}
\@namedef{PY@tok@il}{\def\PY@tc##1{\textcolor[rgb]{0.40,0.40,0.40}{##1}}}
\@namedef{PY@tok@mo}{\def\PY@tc##1{\textcolor[rgb]{0.40,0.40,0.40}{##1}}}
\@namedef{PY@tok@ch}{\let\PY@it=\textit\def\PY@tc##1{\textcolor[rgb]{0.24,0.48,0.48}{##1}}}
\@namedef{PY@tok@cm}{\let\PY@it=\textit\def\PY@tc##1{\textcolor[rgb]{0.24,0.48,0.48}{##1}}}
\@namedef{PY@tok@cpf}{\let\PY@it=\textit\def\PY@tc##1{\textcolor[rgb]{0.24,0.48,0.48}{##1}}}
\@namedef{PY@tok@c1}{\let\PY@it=\textit\def\PY@tc##1{\textcolor[rgb]{0.24,0.48,0.48}{##1}}}
\@namedef{PY@tok@cs}{\let\PY@it=\textit\def\PY@tc##1{\textcolor[rgb]{0.24,0.48,0.48}{##1}}}

\def\PYZbs{\char`\\}
\def\PYZus{\char`\_}
\def\PYZob{\char`\{}
\def\PYZcb{\char`\}}
\def\PYZca{\char`\^}
\def\PYZam{\char`\&}
\def\PYZlt{\char`\<}
\def\PYZgt{\char`\>}
\def\PYZsh{\char`\#}
\def\PYZpc{\char`\%}
\def\PYZdl{\char`\$}
\def\PYZhy{\char`\-}
\def\PYZsq{\char`\'}
\def\PYZdq{\char`\"}
\def\PYZti{\char`\~}
% for compatibility with earlier versions
\def\PYZat{@}
\def\PYZlb{[}
\def\PYZrb{]}
\makeatother


    % For linebreaks inside Verbatim environment from package fancyvrb.
    \makeatletter
        \newbox\Wrappedcontinuationbox
        \newbox\Wrappedvisiblespacebox
        \newcommand*\Wrappedvisiblespace {\textcolor{red}{\textvisiblespace}}
        \newcommand*\Wrappedcontinuationsymbol {\textcolor{red}{\llap{\tiny$\m@th\hookrightarrow$}}}
        \newcommand*\Wrappedcontinuationindent {3ex }
        \newcommand*\Wrappedafterbreak {\kern\Wrappedcontinuationindent\copy\Wrappedcontinuationbox}
        % Take advantage of the already applied Pygments mark-up to insert
        % potential linebreaks for TeX processing.
        %        {, <, #, %, $, ' and ": go to next line.
        %        _, }, ^, &, >, - and ~: stay at end of broken line.
        % Use of \textquotesingle for straight quote.
        \newcommand*\Wrappedbreaksatspecials {%
            \def\PYGZus{\discretionary{\char`\_}{\Wrappedafterbreak}{\char`\_}}%
            \def\PYGZob{\discretionary{}{\Wrappedafterbreak\char`\{}{\char`\{}}%
            \def\PYGZcb{\discretionary{\char`\}}{\Wrappedafterbreak}{\char`\}}}%
            \def\PYGZca{\discretionary{\char`\^}{\Wrappedafterbreak}{\char`\^}}%
            \def\PYGZam{\discretionary{\char`\&}{\Wrappedafterbreak}{\char`\&}}%
            \def\PYGZlt{\discretionary{}{\Wrappedafterbreak\char`\<}{\char`\<}}%
            \def\PYGZgt{\discretionary{\char`\>}{\Wrappedafterbreak}{\char`\>}}%
            \def\PYGZsh{\discretionary{}{\Wrappedafterbreak\char`\#}{\char`\#}}%
            \def\PYGZpc{\discretionary{}{\Wrappedafterbreak\char`\%}{\char`\%}}%
            \def\PYGZdl{\discretionary{}{\Wrappedafterbreak\char`\$}{\char`\$}}%
            \def\PYGZhy{\discretionary{\char`\-}{\Wrappedafterbreak}{\char`\-}}%
            \def\PYGZsq{\discretionary{}{\Wrappedafterbreak\textquotesingle}{\textquotesingle}}%
            \def\PYGZdq{\discretionary{}{\Wrappedafterbreak\char`\"}{\char`\"}}%
            \def\PYGZti{\discretionary{\char`\~}{\Wrappedafterbreak}{\char`\~}}%
        }
        % Some characters . , ; ? ! / are not pygmentized.
        % This macro makes them "active" and they will insert potential linebreaks
        \newcommand*\Wrappedbreaksatpunct {%
            \lccode`\~`\.\lowercase{\def~}{\discretionary{\hbox{\char`\.}}{\Wrappedafterbreak}{\hbox{\char`\.}}}%
            \lccode`\~`\,\lowercase{\def~}{\discretionary{\hbox{\char`\,}}{\Wrappedafterbreak}{\hbox{\char`\,}}}%
            \lccode`\~`\;\lowercase{\def~}{\discretionary{\hbox{\char`\;}}{\Wrappedafterbreak}{\hbox{\char`\;}}}%
            \lccode`\~`\:\lowercase{\def~}{\discretionary{\hbox{\char`\:}}{\Wrappedafterbreak}{\hbox{\char`\:}}}%
            \lccode`\~`\?\lowercase{\def~}{\discretionary{\hbox{\char`\?}}{\Wrappedafterbreak}{\hbox{\char`\?}}}%
            \lccode`\~`\!\lowercase{\def~}{\discretionary{\hbox{\char`\!}}{\Wrappedafterbreak}{\hbox{\char`\!}}}%
            \lccode`\~`\/\lowercase{\def~}{\discretionary{\hbox{\char`\/}}{\Wrappedafterbreak}{\hbox{\char`\/}}}%
            \catcode`\.\active
            \catcode`\,\active
            \catcode`\;\active
            \catcode`\:\active
            \catcode`\?\active
            \catcode`\!\active
            \catcode`\/\active
            \lccode`\~`\~
        }
    \makeatother

    \let\OriginalVerbatim=\Verbatim
    \makeatletter
    \renewcommand{\Verbatim}[1][1]{%
        %\parskip\z@skip
        \sbox\Wrappedcontinuationbox {\Wrappedcontinuationsymbol}%
        \sbox\Wrappedvisiblespacebox {\FV@SetupFont\Wrappedvisiblespace}%
        \def\FancyVerbFormatLine ##1{\hsize\linewidth
            \vtop{\raggedright\hyphenpenalty\z@\exhyphenpenalty\z@
                \doublehyphendemerits\z@\finalhyphendemerits\z@
                \strut ##1\strut}%
        }%
        % If the linebreak is at a space, the latter will be displayed as visible
        % space at end of first line, and a continuation symbol starts next line.
        % Stretch/shrink are however usually zero for typewriter font.
        \def\FV@Space {%
            \nobreak\hskip\z@ plus\fontdimen3\font minus\fontdimen4\font
            \discretionary{\copy\Wrappedvisiblespacebox}{\Wrappedafterbreak}
            {\kern\fontdimen2\font}%
        }%

        % Allow breaks at special characters using \PYG... macros.
        \Wrappedbreaksatspecials
        % Breaks at punctuation characters . , ; ? ! and / need catcode=\active
        \OriginalVerbatim[#1,codes*=\Wrappedbreaksatpunct]%
    }
    \makeatother

    % Exact colors from NB
    \definecolor{incolor}{HTML}{303F9F}
    \definecolor{outcolor}{HTML}{D84315}
    \definecolor{cellborder}{HTML}{CFCFCF}
    \definecolor{cellbackground}{HTML}{F7F7F7}

    % prompt
    \makeatletter
    \newcommand{\boxspacing}{\kern\kvtcb@left@rule\kern\kvtcb@boxsep}
    \makeatother
    \newcommand{\prompt}[4]{
        {\ttfamily\llap{{\color{#2}[#3]:\hspace{3pt}#4}}\vspace{-\baselineskip}}
    }
    

    
    % Prevent overflowing lines due to hard-to-break entities
    \sloppy
    % Setup hyperref package
    \hypersetup{
      breaklinks=true,  % so long urls are correctly broken across lines
      colorlinks=true,
      urlcolor=urlcolor,
      linkcolor=linkcolor,
      citecolor=citecolor,
      }
    % Slightly bigger margins than the latex defaults
    
    \geometry{verbose,tmargin=1in,bmargin=1in,lmargin=1in,rmargin=1in}
    
    

\begin{document}
    
    \maketitle
    
    

    
    \hypertarget{exercuxedcio-1}{%
\section{Exercício 1}\label{exercuxedcio-1}}

    Façamos nossa função de decomposição \(QR\) com Gram-Schmidt.

    \begin{tcolorbox}[breakable, size=fbox, boxrule=1pt, pad at break*=1mm,colback=cellbackground, colframe=cellborder]
\prompt{In}{incolor}{88}{\boxspacing}
\begin{Verbatim}[commandchars=\\\{\}]
\PY{c+c1}{// Função de decomposição QR de Gram\PYZhy{}Schmidt}
\PY{k}{function}\PY{+w}{ }\PY{n+nf}{[Q, R] = qr\PYZus{}GS}\PY{p}{(}A\PY{p}{)}
\PY{+w}{    }\PY{c+c1}{// Pegando as dimensões da A}
\PY{+w}{    }\PY{p}{[}\PY{n}{m}\PY{p}{,}\PY{+w}{ }\PY{n}{n}\PY{p}{]}\PY{+w}{ }\PY{p}{=}\PY{+w}{ }\PY{n+nb}{size}\PY{p}{(}\PY{n}{A}\PY{p}{)}
\PY{+w}{    }\PY{c+c1}{// Inicializando as matrizes Q e R}
\PY{+w}{    }\PY{n}{Q}\PY{+w}{ }\PY{p}{=}\PY{+w}{ }\PY{n+nb}{zeros}\PY{p}{(}\PY{n}{m}\PY{p}{,}\PY{+w}{ }\PY{n}{n}\PY{p}{)}
\PY{+w}{    }\PY{n}{R}\PY{+w}{ }\PY{p}{=}\PY{+w}{ }\PY{n+nb}{zeros}\PY{p}{(}\PY{n}{n}\PY{p}{,}\PY{+w}{ }\PY{n}{n}\PY{p}{)}

\PY{+w}{    }\PY{c+c1}{// Para cada coluna...}
\PY{+w}{    }\PY{k}{for}\PY{+w}{ }\PY{n}{j}\PY{+w}{ }\PY{p}{=}\PY{+w}{ }\PY{l+m+mi}{1}\PY{p}{:}\PY{n}{n}
\PY{+w}{        }\PY{c+c1}{// Inicializa o vetor v como a coluna de A}
\PY{+w}{        }\PY{n}{v}\PY{+w}{ }\PY{p}{=}\PY{+w}{ }\PY{n}{A}\PY{p}{(:,}\PY{+w}{ }\PY{n}{j}\PY{p}{)}
\PY{+w}{        }\PY{c+c1}{// Se não for a primeira coluna...}
\PY{+w}{        }\PY{k}{if}\PY{+w}{ }\PY{n}{j}\PY{+w}{ }\PY{o}{\PYZgt{}}\PY{+w}{ }\PY{l+m+mi}{1}\PY{+w}{ }\PY{n+nb}{then}
\PY{+w}{            }\PY{c+c1}{// Calcula o termo de projeção de v}
\PY{+w}{            }\PY{n}{R}\PY{p}{(}\PY{l+m+mi}{1}\PY{p}{:(}\PY{n}{j}\PY{o}{\PYZhy{}}\PY{l+m+mi}{1}\PY{p}{),}\PY{+w}{ }\PY{n}{j}\PY{p}{)}\PY{+w}{ }\PY{p}{=}\PY{+w}{ }\PY{p}{(}\PY{n}{Q}\PY{p}{(:,}\PY{+w}{ }\PY{l+m+mi}{1}\PY{p}{:(}\PY{n}{j}\PY{o}{\PYZhy{}}\PY{l+m+mi}{1}\PY{p}{)))}\PY{o}{\PYZsq{}}\PY{+w}{ }\PY{o}{*}\PY{+w}{ }\PY{n}{A}\PY{p}{(:,}\PY{+w}{ }\PY{n}{j}\PY{p}{)}
\PY{+w}{            }\PY{c+c1}{// Calcula a ortogonalização de v}
\PY{+w}{            }\PY{n}{v}\PY{+w}{ }\PY{p}{=}\PY{+w}{ }\PY{n}{v}\PY{+w}{ }\PY{o}{\PYZhy{}}\PY{+w}{ }\PY{n}{Q}\PY{p}{(:,}\PY{+w}{ }\PY{l+m+mi}{1}\PY{p}{:(}\PY{n}{j}\PY{o}{\PYZhy{}}\PY{l+m+mi}{1}\PY{p}{))}\PY{+w}{ }\PY{o}{*}\PY{+w}{ }\PY{n}{R}\PY{p}{(}\PY{l+m+mi}{1}\PY{p}{:(}\PY{n}{j}\PY{o}{\PYZhy{}}\PY{l+m+mi}{1}\PY{p}{),}\PY{+w}{ }\PY{n}{j}\PY{p}{)}
\PY{+w}{        }\PY{k}{end}
\PY{+w}{        }\PY{c+c1}{// Salva o termo da diagonal de R como a norma de v}
\PY{+w}{        }\PY{n}{R}\PY{p}{(}\PY{n}{j}\PY{p}{,}\PY{+w}{ }\PY{n}{j}\PY{p}{)}\PY{+w}{ }\PY{p}{=}\PY{+w}{ }\PY{n+nb}{norm}\PY{p}{(}\PY{n}{v}\PY{p}{)}
\PY{+w}{        }\PY{c+c1}{// Salva a coluna de Q como o vetor v normalizado}
\PY{+w}{        }\PY{n}{Q}\PY{p}{(:,}\PY{+w}{ }\PY{n}{j}\PY{p}{)}\PY{+w}{ }\PY{p}{=}\PY{+w}{ }\PY{n}{v}\PY{+w}{ }\PY{o}{/}\PY{+w}{ }\PY{n}{R}\PY{p}{(}\PY{n}{j}\PY{p}{,}\PY{+w}{ }\PY{n}{j}\PY{p}{)}
\PY{+w}{    }\PY{k}{end}
\PY{k}{endfunction}
\end{Verbatim}
\end{tcolorbox}

    Vamos testá-la com algumas matrizes retangulares, calculando os erros do
produto \(QR\) com a matriz \(A\) e da ortogonalidade da matriz \(Q\).

    \begin{tcolorbox}[breakable, size=fbox, boxrule=1pt, pad at break*=1mm,colback=cellbackground, colframe=cellborder]
\prompt{In}{incolor}{89}{\boxspacing}
\begin{Verbatim}[commandchars=\\\{\}]
\PY{c+c1}{// Gerando uma matriz}
\PY{n}{A1}\PY{+w}{ }\PY{p}{=}\PY{+w}{ }\PY{n+nb}{round}\PY{p}{(}\PY{l+m+mi}{20}\PY{o}{*}\PY{n+nb}{rand}\PY{p}{(}\PY{l+m+mi}{3}\PY{p}{,}\PY{+w}{ }\PY{l+m+mi}{2}\PY{p}{)}\PY{+w}{ }\PY{o}{\PYZhy{}}\PY{+w}{ }\PY{l+m+mi}{10}\PY{p}{)}
\PY{c+c1}{// Calculando a decomposição QR}
\PY{p}{[}\PY{n}{Q1}\PY{p}{,}\PY{+w}{ }\PY{n}{R1}\PY{p}{]}\PY{+w}{ }\PY{p}{=}\PY{+w}{ }\PY{n}{qr\PYZus{}GS}\PY{p}{(}\PY{n}{A1}\PY{p}{);}
\PY{n+nb}{disp}\PY{p}{(}\PY{l+s}{\PYZdq{}norm(Q*R \PYZhy{} A)\PYZdq{}}\PY{p}{,}\PY{+w}{ }\PY{n+nb}{norm}\PY{p}{(}\PY{n}{Q1}\PY{o}{*}\PY{n}{R1}\PY{+w}{ }\PY{o}{\PYZhy{}}\PY{+w}{ }\PY{n}{A1}\PY{p}{))}
\PY{n+nb}{disp}\PY{p}{(}\PY{l+s}{\PYZdq{}norm(Q\PYZca{}T*Q \PYZhy{} I)\PYZdq{}}\PY{p}{,}\PY{+w}{ }\PY{n+nb}{norm}\PY{p}{(}\PY{n}{Q1}\PY{o}{\PYZsq{}}\PY{o}{*}\PY{n}{Q1}\PY{+w}{ }\PY{o}{\PYZhy{}}\PY{+w}{ }\PY{n+nb}{eye}\PY{p}{(}\PY{l+m+mi}{2}\PY{p}{,}\PY{+w}{ }\PY{l+m+mi}{2}\PY{p}{)))}
\end{Verbatim}
\end{tcolorbox}

    \begin{Verbatim}[commandchars=\\\{\}]
 A1  =
   2.   9.
  -2.   7.
   4.  -3.

  "norm(Q*R - A)"

   1.776D-15

  "norm(Q\^{}T*Q - I)"

   2.292D-16
    \end{Verbatim}

    \begin{tcolorbox}[breakable, size=fbox, boxrule=1pt, pad at break*=1mm,colback=cellbackground, colframe=cellborder]
\prompt{In}{incolor}{90}{\boxspacing}
\begin{Verbatim}[commandchars=\\\{\}]
\PY{n}{A2}\PY{+w}{ }\PY{p}{=}\PY{+w}{ }\PY{n+nb}{round}\PY{p}{(}\PY{l+m+mi}{20}\PY{o}{*}\PY{n+nb}{rand}\PY{p}{(}\PY{l+m+mi}{4}\PY{p}{,}\PY{+w}{ }\PY{l+m+mi}{3}\PY{p}{)}\PY{+w}{ }\PY{o}{\PYZhy{}}\PY{+w}{ }\PY{l+m+mi}{10}\PY{p}{)}
\PY{p}{[}\PY{n}{Q2}\PY{p}{,}\PY{+w}{ }\PY{n}{R2}\PY{p}{]}\PY{+w}{ }\PY{p}{=}\PY{+w}{ }\PY{n}{qr\PYZus{}GS}\PY{p}{(}\PY{n}{A2}\PY{p}{);}
\PY{n+nb}{disp}\PY{p}{(}\PY{l+s}{\PYZdq{}norm(Q*R \PYZhy{} A)\PYZdq{}}\PY{p}{,}\PY{+w}{ }\PY{n+nb}{norm}\PY{p}{(}\PY{n}{Q2}\PY{o}{*}\PY{n}{R2}\PY{+w}{ }\PY{o}{\PYZhy{}}\PY{+w}{ }\PY{n}{A2}\PY{p}{))}
\PY{n+nb}{disp}\PY{p}{(}\PY{l+s}{\PYZdq{}norm(Q\PYZca{}T*Q \PYZhy{} I)\PYZdq{}}\PY{p}{,}\PY{+w}{ }\PY{n+nb}{norm}\PY{p}{(}\PY{n}{Q2}\PY{o}{\PYZsq{}}\PY{o}{*}\PY{n}{Q2}\PY{+w}{ }\PY{o}{\PYZhy{}}\PY{+w}{ }\PY{n+nb}{eye}\PY{p}{(}\PY{l+m+mi}{3}\PY{p}{,}\PY{+w}{ }\PY{l+m+mi}{3}\PY{p}{)))}
\end{Verbatim}
\end{tcolorbox}

    \begin{Verbatim}[commandchars=\\\{\}]
 A2  =
   8.   9.    5.
  -8.   10.   1.
   1.   2.    5.
   1.   10.  -9.

  "norm(Q*R - A)"

   1.891D-15

  "norm(Q\^{}T*Q - I)"

   2.562D-16
    \end{Verbatim}

    \begin{tcolorbox}[breakable, size=fbox, boxrule=1pt, pad at break*=1mm,colback=cellbackground, colframe=cellborder]
\prompt{In}{incolor}{91}{\boxspacing}
\begin{Verbatim}[commandchars=\\\{\}]
\PY{n}{A3}\PY{+w}{ }\PY{p}{=}\PY{+w}{ }\PY{n+nb}{round}\PY{p}{(}\PY{l+m+mi}{20}\PY{o}{*}\PY{n+nb}{rand}\PY{p}{(}\PY{l+m+mi}{5}\PY{p}{,}\PY{+w}{ }\PY{l+m+mi}{5}\PY{p}{)}\PY{+w}{ }\PY{o}{\PYZhy{}}\PY{+w}{ }\PY{l+m+mi}{10}\PY{p}{)}
\PY{p}{[}\PY{n}{Q3}\PY{p}{,}\PY{+w}{ }\PY{n}{R3}\PY{p}{]}\PY{+w}{ }\PY{p}{=}\PY{+w}{ }\PY{n}{qr\PYZus{}GS}\PY{p}{(}\PY{n}{A3}\PY{p}{);}
\PY{n+nb}{disp}\PY{p}{(}\PY{l+s}{\PYZdq{}norm(Q*R \PYZhy{} A)\PYZdq{}}\PY{p}{,}\PY{+w}{ }\PY{n+nb}{norm}\PY{p}{(}\PY{n}{Q3}\PY{o}{*}\PY{n}{R3}\PY{+w}{ }\PY{o}{\PYZhy{}}\PY{+w}{ }\PY{n}{A3}\PY{p}{))}
\PY{n+nb}{disp}\PY{p}{(}\PY{l+s}{\PYZdq{}norm(Q\PYZca{}T*Q \PYZhy{} I)\PYZdq{}}\PY{p}{,}\PY{+w}{ }\PY{n+nb}{norm}\PY{p}{(}\PY{n}{Q3}\PY{o}{\PYZsq{}}\PY{o}{*}\PY{n}{Q3}\PY{+w}{ }\PY{o}{\PYZhy{}}\PY{+w}{ }\PY{n+nb}{eye}\PY{p}{(}\PY{l+m+mi}{5}\PY{p}{,}\PY{+w}{ }\PY{l+m+mi}{5}\PY{p}{)))}
\end{Verbatim}
\end{tcolorbox}

    \begin{Verbatim}[commandchars=\\\{\}]
 A3  =
  -2.   8.   6.  -5.   4.
  -8.   5.   4.   4.  -4.
  -5.   1.  -8.   9.  -7.
  -3.  -2.  -5.   0.   4.
  -7.   8.   5.  -5.  -1.

  "norm(Q*R - A)"

   8.882D-16

  "norm(Q\^{}T*Q - I)"

   6.279D-16
    \end{Verbatim}

    Parece que nossa função está funcionando bem.

\hypertarget{exercuxedcio-2}{%
\section{Exercício 2}\label{exercuxedcio-2}}

    Agora, façamos a função da decomposição \(QR\) com Gram-Schmidt
modificado. Infelizmente, para essa versão, não é possível calcular o
loop interno matricialmente devido à atualização do vetor \(v\), sendo
necessário um loop \texttt{for}.

    \begin{tcolorbox}[breakable, size=fbox, boxrule=1pt, pad at break*=1mm,colback=cellbackground, colframe=cellborder]
\prompt{In}{incolor}{92}{\boxspacing}
\begin{Verbatim}[commandchars=\\\{\}]
\PY{c+c1}{// Função de decomposição QR de Gram\PYZhy{}Schmidt}
\PY{k}{function}\PY{+w}{ }\PY{n+nf}{[Q, R] = qr\PYZus{}GSM}\PY{p}{(}A\PY{p}{)}
\PY{+w}{    }\PY{c+c1}{// Pegando as dimensões da A}
\PY{+w}{    }\PY{p}{[}\PY{n}{m}\PY{p}{,}\PY{+w}{ }\PY{n}{n}\PY{p}{]}\PY{+w}{ }\PY{p}{=}\PY{+w}{ }\PY{n+nb}{size}\PY{p}{(}\PY{n}{A}\PY{p}{)}
\PY{+w}{    }\PY{c+c1}{// Inicializando as matrizes Q e R}
\PY{+w}{    }\PY{n}{Q}\PY{+w}{ }\PY{p}{=}\PY{+w}{ }\PY{n+nb}{zeros}\PY{p}{(}\PY{n}{m}\PY{p}{,}\PY{+w}{ }\PY{n}{n}\PY{p}{)}
\PY{+w}{    }\PY{n}{R}\PY{+w}{ }\PY{p}{=}\PY{+w}{ }\PY{n+nb}{zeros}\PY{p}{(}\PY{n}{n}\PY{p}{,}\PY{+w}{ }\PY{n}{n}\PY{p}{)}

\PY{+w}{    }\PY{c+c1}{// Para cada coluna...}
\PY{+w}{    }\PY{k}{for}\PY{+w}{ }\PY{n}{j}\PY{+w}{ }\PY{p}{=}\PY{+w}{ }\PY{l+m+mi}{1}\PY{p}{:}\PY{n}{n}
\PY{+w}{        }\PY{c+c1}{// Inicializa o vetor v como a coluna de A}
\PY{+w}{        }\PY{n}{v}\PY{+w}{ }\PY{p}{=}\PY{+w}{ }\PY{n}{A}\PY{p}{(:,}\PY{+w}{ }\PY{n}{j}\PY{p}{)}
\PY{+w}{        }\PY{c+c1}{// Para cada linha...}
\PY{+w}{        }\PY{k}{for}\PY{+w}{ }\PY{n}{i}\PY{+w}{ }\PY{p}{=}\PY{+w}{ }\PY{l+m+mi}{1}\PY{p}{:(}\PY{n}{j}\PY{o}{\PYZhy{}}\PY{l+m+mi}{1}\PY{p}{)}
\PY{+w}{            }\PY{c+c1}{// Calcula o fator de projeção do v sobre qi}
\PY{+w}{            }\PY{n}{R}\PY{p}{(}\PY{n}{i}\PY{p}{,}\PY{+w}{ }\PY{n}{j}\PY{p}{)}\PY{+w}{ }\PY{p}{=}\PY{+w}{ }\PY{n}{Q}\PY{p}{(:,}\PY{+w}{ }\PY{n}{i}\PY{p}{)}\PY{o}{\PYZsq{}}\PY{+w}{ }\PY{o}{*}\PY{+w}{ }\PY{n}{v}
\PY{+w}{            }\PY{c+c1}{// Ortogonaliza em relação ao qi}
\PY{+w}{            }\PY{n}{v}\PY{+w}{ }\PY{p}{=}\PY{+w}{ }\PY{n}{v}\PY{+w}{ }\PY{o}{\PYZhy{}}\PY{+w}{ }\PY{n}{R}\PY{p}{(}\PY{n}{i}\PY{p}{,}\PY{+w}{ }\PY{n}{j}\PY{p}{)}\PY{+w}{ }\PY{o}{*}\PY{+w}{ }\PY{n}{Q}\PY{p}{(:,}\PY{+w}{ }\PY{n}{i}\PY{p}{)}
\PY{+w}{        }\PY{k}{end}
\PY{+w}{        }\PY{c+c1}{// Salva o termo da diagonal de R como a norma de v}
\PY{+w}{        }\PY{n}{R}\PY{p}{(}\PY{n}{j}\PY{p}{,}\PY{+w}{ }\PY{n}{j}\PY{p}{)}\PY{+w}{ }\PY{p}{=}\PY{+w}{ }\PY{n+nb}{norm}\PY{p}{(}\PY{n}{v}\PY{p}{)}
\PY{+w}{        }\PY{c+c1}{// Salva a coluna de Q como o vetor v normalizado}
\PY{+w}{        }\PY{n}{Q}\PY{p}{(:,}\PY{+w}{ }\PY{n}{j}\PY{p}{)}\PY{+w}{ }\PY{p}{=}\PY{+w}{ }\PY{n}{v}\PY{+w}{ }\PY{o}{/}\PY{+w}{ }\PY{n}{R}\PY{p}{(}\PY{n}{j}\PY{p}{,}\PY{+w}{ }\PY{n}{j}\PY{p}{)}
\PY{+w}{    }\PY{k}{end}
\PY{k}{endfunction}
\end{Verbatim}
\end{tcolorbox}

    Vamos agora testá-la com as mesmas matrizes usadas no exercício
anterior, comparando a precisão dos dois algoritmos.

    \begin{tcolorbox}[breakable, size=fbox, boxrule=1pt, pad at break*=1mm,colback=cellbackground, colframe=cellborder]
\prompt{In}{incolor}{93}{\boxspacing}
\begin{Verbatim}[commandchars=\\\{\}]
\PY{c+c1}{// Calculando a decomposição QR pelo GS modificado}
\PY{p}{[}\PY{n}{QM1}\PY{p}{,}\PY{+w}{ }\PY{n}{RM1}\PY{p}{]}\PY{+w}{ }\PY{p}{=}\PY{+w}{ }\PY{n}{qr\PYZus{}GSM}\PY{p}{(}\PY{n}{A1}\PY{p}{);}

\PY{n+nb}{disp}\PY{p}{(}\PY{l+s}{\PYZdq{}=============== Gram\PYZhy{}Schmidt ================\PYZdq{}}\PY{p}{)}
\PY{n+nb}{disp}\PY{p}{(}\PY{l+s}{\PYZdq{}norm(Q*R \PYZhy{} A)\PYZdq{}}\PY{p}{,}\PY{+w}{ }\PY{n+nb}{norm}\PY{p}{(}\PY{n}{Q1}\PY{o}{*}\PY{n}{R1}\PY{+w}{ }\PY{o}{\PYZhy{}}\PY{+w}{ }\PY{n}{A1}\PY{p}{))}
\PY{n+nb}{disp}\PY{p}{(}\PY{l+s}{\PYZdq{}norm(Q\PYZca{}T*Q \PYZhy{} I)\PYZdq{}}\PY{p}{,}\PY{+w}{ }\PY{n+nb}{norm}\PY{p}{(}\PY{n}{Q1}\PY{o}{\PYZsq{}}\PY{o}{*}\PY{n}{Q1}\PY{+w}{ }\PY{o}{\PYZhy{}}\PY{+w}{ }\PY{n+nb}{eye}\PY{p}{(}\PY{l+m+mi}{2}\PY{p}{,}\PY{+w}{ }\PY{l+m+mi}{2}\PY{p}{)))}
\PY{n+nb}{disp}\PY{p}{(}\PY{l+s}{\PYZdq{}========== Gram\PYZhy{}Schmidt modificado ==========\PYZdq{}}\PY{p}{)}
\PY{n+nb}{disp}\PY{p}{(}\PY{l+s}{\PYZdq{}norm(Q*R \PYZhy{} A)\PYZdq{}}\PY{p}{,}\PY{+w}{ }\PY{n+nb}{norm}\PY{p}{(}\PY{n}{QM1}\PY{o}{*}\PY{n}{RM1}\PY{+w}{ }\PY{o}{\PYZhy{}}\PY{+w}{ }\PY{n}{A1}\PY{p}{))}
\PY{n+nb}{disp}\PY{p}{(}\PY{l+s}{\PYZdq{}norm(Q\PYZca{}T*Q \PYZhy{} I)\PYZdq{}}\PY{p}{,}\PY{+w}{ }\PY{n+nb}{norm}\PY{p}{(}\PY{n}{QM1}\PY{o}{\PYZsq{}}\PY{o}{*}\PY{n}{QM1}\PY{+w}{ }\PY{o}{\PYZhy{}}\PY{+w}{ }\PY{n+nb}{eye}\PY{p}{(}\PY{l+m+mi}{2}\PY{p}{,}\PY{+w}{ }\PY{l+m+mi}{2}\PY{p}{)))}
\end{Verbatim}
\end{tcolorbox}

    \begin{Verbatim}[commandchars=\\\{\}]
  "=============== Gram-Schmidt ================"

  "norm(Q*R - A)"

   1.776D-15

  "norm(Q\^{}T*Q - I)"

   2.292D-16

  "========== Gram-Schmidt modificado =========="

  "norm(Q*R - A)"

   1.776D-15

  "norm(Q\^{}T*Q - I)"

   2.292D-16
    \end{Verbatim}

    \begin{tcolorbox}[breakable, size=fbox, boxrule=1pt, pad at break*=1mm,colback=cellbackground, colframe=cellborder]
\prompt{In}{incolor}{94}{\boxspacing}
\begin{Verbatim}[commandchars=\\\{\}]
\PY{p}{[}\PY{n}{QM2}\PY{p}{,}\PY{+w}{ }\PY{n}{RM2}\PY{p}{]}\PY{+w}{ }\PY{p}{=}\PY{+w}{ }\PY{n}{qr\PYZus{}GSM}\PY{p}{(}\PY{n}{A2}\PY{p}{);}

\PY{n+nb}{disp}\PY{p}{(}\PY{l+s}{\PYZdq{}=============== Gram\PYZhy{}Schmidt ================\PYZdq{}}\PY{p}{)}
\PY{n+nb}{disp}\PY{p}{(}\PY{l+s}{\PYZdq{}norm(Q*R \PYZhy{} A)\PYZdq{}}\PY{p}{,}\PY{+w}{ }\PY{n+nb}{norm}\PY{p}{(}\PY{n}{Q2}\PY{o}{*}\PY{n}{R2}\PY{+w}{ }\PY{o}{\PYZhy{}}\PY{+w}{ }\PY{n}{A2}\PY{p}{))}
\PY{n+nb}{disp}\PY{p}{(}\PY{l+s}{\PYZdq{}norm(Q\PYZca{}T*Q \PYZhy{} I)\PYZdq{}}\PY{p}{,}\PY{+w}{ }\PY{n+nb}{norm}\PY{p}{(}\PY{n}{Q2}\PY{o}{\PYZsq{}}\PY{o}{*}\PY{n}{Q2}\PY{+w}{ }\PY{o}{\PYZhy{}}\PY{+w}{ }\PY{n+nb}{eye}\PY{p}{(}\PY{l+m+mi}{3}\PY{p}{,}\PY{+w}{ }\PY{l+m+mi}{3}\PY{p}{)))}
\PY{n+nb}{disp}\PY{p}{(}\PY{l+s}{\PYZdq{}========== Gram\PYZhy{}Schmidt modificado ==========\PYZdq{}}\PY{p}{)}
\PY{n+nb}{disp}\PY{p}{(}\PY{l+s}{\PYZdq{}norm(Q*R \PYZhy{} A)\PYZdq{}}\PY{p}{,}\PY{+w}{ }\PY{n+nb}{norm}\PY{p}{(}\PY{n}{QM2}\PY{o}{*}\PY{n}{RM2}\PY{+w}{ }\PY{o}{\PYZhy{}}\PY{+w}{ }\PY{n}{A2}\PY{p}{))}
\PY{n+nb}{disp}\PY{p}{(}\PY{l+s}{\PYZdq{}norm(Q\PYZca{}T*Q \PYZhy{} I)\PYZdq{}}\PY{p}{,}\PY{+w}{ }\PY{n+nb}{norm}\PY{p}{(}\PY{n}{QM2}\PY{o}{\PYZsq{}}\PY{o}{*}\PY{n}{QM2}\PY{+w}{ }\PY{o}{\PYZhy{}}\PY{+w}{ }\PY{n+nb}{eye}\PY{p}{(}\PY{l+m+mi}{3}\PY{p}{,}\PY{+w}{ }\PY{l+m+mi}{3}\PY{p}{)))}
\end{Verbatim}
\end{tcolorbox}

    \begin{Verbatim}[commandchars=\\\{\}]
  "=============== Gram-Schmidt ================"

  "norm(Q*R - A)"

   1.891D-15

  "norm(Q\^{}T*Q - I)"

   2.562D-16

  "========== Gram-Schmidt modificado =========="

  "norm(Q*R - A)"

   1.891D-15

  "norm(Q\^{}T*Q - I)"

   2.562D-16
    \end{Verbatim}

    \begin{tcolorbox}[breakable, size=fbox, boxrule=1pt, pad at break*=1mm,colback=cellbackground, colframe=cellborder]
\prompt{In}{incolor}{95}{\boxspacing}
\begin{Verbatim}[commandchars=\\\{\}]
\PY{p}{[}\PY{n}{QM3}\PY{p}{,}\PY{+w}{ }\PY{n}{RM3}\PY{p}{]}\PY{+w}{ }\PY{p}{=}\PY{+w}{ }\PY{n}{qr\PYZus{}GSM}\PY{p}{(}\PY{n}{A3}\PY{p}{);}

\PY{n+nb}{disp}\PY{p}{(}\PY{l+s}{\PYZdq{}=============== Gram\PYZhy{}Schmidt ================\PYZdq{}}\PY{p}{)}
\PY{n+nb}{disp}\PY{p}{(}\PY{l+s}{\PYZdq{}norm(Q*R \PYZhy{} A)\PYZdq{}}\PY{p}{,}\PY{+w}{ }\PY{n+nb}{norm}\PY{p}{(}\PY{n}{Q3}\PY{o}{*}\PY{n}{R3}\PY{+w}{ }\PY{o}{\PYZhy{}}\PY{+w}{ }\PY{n}{A3}\PY{p}{))}
\PY{n+nb}{disp}\PY{p}{(}\PY{l+s}{\PYZdq{}norm(Q\PYZca{}T*Q \PYZhy{} I)\PYZdq{}}\PY{p}{,}\PY{+w}{ }\PY{n+nb}{norm}\PY{p}{(}\PY{n}{Q3}\PY{o}{\PYZsq{}}\PY{o}{*}\PY{n}{Q3}\PY{+w}{ }\PY{o}{\PYZhy{}}\PY{+w}{ }\PY{n+nb}{eye}\PY{p}{(}\PY{l+m+mi}{5}\PY{p}{,}\PY{+w}{ }\PY{l+m+mi}{5}\PY{p}{)))}
\PY{n+nb}{disp}\PY{p}{(}\PY{l+s}{\PYZdq{}========== Gram\PYZhy{}Schmidt modificado ==========\PYZdq{}}\PY{p}{)}
\PY{n+nb}{disp}\PY{p}{(}\PY{l+s}{\PYZdq{}norm(Q*R \PYZhy{} A)\PYZdq{}}\PY{p}{,}\PY{+w}{ }\PY{n+nb}{norm}\PY{p}{(}\PY{n}{QM3}\PY{o}{*}\PY{n}{RM3}\PY{+w}{ }\PY{o}{\PYZhy{}}\PY{+w}{ }\PY{n}{A3}\PY{p}{))}
\PY{n+nb}{disp}\PY{p}{(}\PY{l+s}{\PYZdq{}norm(Q\PYZca{}T*Q \PYZhy{} I)\PYZdq{}}\PY{p}{,}\PY{+w}{ }\PY{n+nb}{norm}\PY{p}{(}\PY{n}{QM3}\PY{o}{\PYZsq{}}\PY{o}{*}\PY{n}{QM3}\PY{+w}{ }\PY{o}{\PYZhy{}}\PY{+w}{ }\PY{n+nb}{eye}\PY{p}{(}\PY{l+m+mi}{5}\PY{p}{,}\PY{+w}{ }\PY{l+m+mi}{5}\PY{p}{)))}
\end{Verbatim}
\end{tcolorbox}

    \begin{Verbatim}[commandchars=\\\{\}]
  "=============== Gram-Schmidt ================"

  "norm(Q*R - A)"

   8.882D-16

  "norm(Q\^{}T*Q - I)"

   6.279D-16

  "========== Gram-Schmidt modificado =========="

  "norm(Q*R - A)"

   1.776D-15

  "norm(Q\^{}T*Q - I)"

   5.514D-16
    \end{Verbatim}

    Nota-se que, para matrizes comuns, os algoritmos obtiveram desempenhos
bem similares, não variando muito o erro numérico tanto da aproximação
da \(QR\) para a \(A\) quanto da ortogonalidade da \(Q\).

    \hypertarget{exercuxedcio-4}{%
\section{Exercício 4}\label{exercuxedcio-4}}

    Agora, vamos escrever funções para calcular a decomposição \(QR\) de uma
matriz com base no refletor de Householder. Primeiro, temos a versão
normal do algoritmo escrita abaixo.

    \begin{tcolorbox}[breakable, size=fbox, boxrule=1pt, pad at break*=1mm,colback=cellbackground, colframe=cellborder]
\prompt{In}{incolor}{96}{\boxspacing}
\begin{Verbatim}[commandchars=\\\{\}]
\PY{c+c1}{// Função da versão 1 do algoritmo de Householder para decomposição QR}
\PY{k}{function}\PY{+w}{ }\PY{n+nf}{[U, R] = qr\PYZus{}House\PYZus{}v1}\PY{p}{(}A\PY{p}{)}
\PY{+w}{    }\PY{c+c1}{// Pegando as dimensões da A}
\PY{+w}{    }\PY{p}{[}\PY{n}{m}\PY{p}{,}\PY{+w}{ }\PY{n}{n}\PY{p}{]}\PY{+w}{ }\PY{p}{=}\PY{+w}{ }\PY{n+nb}{size}\PY{p}{(}\PY{n}{A}\PY{p}{)}
\PY{+w}{    }\PY{c+c1}{// Inicializando a matriz U}
\PY{+w}{    }\PY{n}{U}\PY{+w}{ }\PY{p}{=}\PY{+w}{ }\PY{n+nb}{zeros}\PY{p}{(}\PY{n}{m}\PY{p}{,}\PY{+w}{ }\PY{n}{n}\PY{p}{)}

\PY{+w}{    }\PY{c+c1}{// Para cada coluna...}
\PY{+w}{    }\PY{k}{for}\PY{+w}{ }\PY{n}{j}\PY{+w}{ }\PY{p}{=}\PY{+w}{ }\PY{l+m+mi}{1}\PY{p}{:}\PY{n}{n}
\PY{+w}{        }\PY{c+c1}{// Pega o vetor que queremos projetar sobre um eixo}
\PY{+w}{        }\PY{n}{x}\PY{+w}{ }\PY{p}{=}\PY{+w}{ }\PY{n}{A}\PY{p}{(}\PY{n}{j}\PY{p}{:}\PY{n}{m}\PY{p}{,}\PY{+w}{ }\PY{n}{j}\PY{p}{)}
\PY{+w}{        }\PY{c+c1}{// Escolhe a melhor das duas projeções possíveis}
\PY{+w}{        }\PY{k}{if}\PY{+w}{ }\PY{n}{x}\PY{p}{(}\PY{l+m+mi}{1}\PY{p}{)}\PY{+w}{ }\PY{o}{\PYZgt{}}\PY{+w}{ }\PY{l+m+mi}{0}\PY{+w}{ }\PY{n+nb}{then}
\PY{+w}{            }\PY{n}{x}\PY{p}{(}\PY{l+m+mi}{1}\PY{p}{)}\PY{+w}{ }\PY{p}{=}\PY{+w}{ }\PY{n}{x}\PY{p}{(}\PY{l+m+mi}{1}\PY{p}{)}\PY{+w}{ }\PY{o}{+}\PY{+w}{ }\PY{n+nb}{norm}\PY{p}{(}\PY{n}{x}\PY{p}{)}
\PY{+w}{        }\PY{k}{else}
\PY{+w}{            }\PY{n}{x}\PY{p}{(}\PY{l+m+mi}{1}\PY{p}{)}\PY{+w}{ }\PY{p}{=}\PY{+w}{ }\PY{n}{x}\PY{p}{(}\PY{l+m+mi}{1}\PY{p}{)}\PY{+w}{ }\PY{o}{\PYZhy{}}\PY{+w}{ }\PY{n+nb}{norm}\PY{p}{(}\PY{n}{x}\PY{p}{)}
\PY{+w}{        }\PY{k}{end}
\PY{+w}{        }\PY{c+c1}{// Calcula u normalizando x}
\PY{+w}{        }\PY{n}{u}\PY{+w}{ }\PY{p}{=}\PY{+w}{ }\PY{n}{x}\PY{o}{/}\PY{n+nb}{norm}\PY{p}{(}\PY{n}{x}\PY{p}{)}
\PY{+w}{        }\PY{c+c1}{// Salva u na matriz U}
\PY{+w}{        }\PY{n}{U}\PY{p}{(}\PY{n}{j}\PY{p}{:}\PY{n}{m}\PY{p}{,}\PY{+w}{ }\PY{n}{j}\PY{p}{)}\PY{+w}{ }\PY{p}{=}\PY{+w}{ }\PY{n}{u}
\PY{+w}{        }\PY{c+c1}{// Triangulariza a parte da matriz correspondente}
\PY{+w}{        }\PY{n}{A}\PY{p}{(}\PY{n}{j}\PY{p}{:}\PY{n}{m}\PY{p}{,}\PY{+w}{ }\PY{n}{j}\PY{p}{:}\PY{n}{n}\PY{p}{)}\PY{+w}{ }\PY{p}{=}\PY{+w}{ }\PY{n}{A}\PY{p}{(}\PY{n}{j}\PY{p}{:}\PY{n}{m}\PY{p}{,}\PY{+w}{ }\PY{n}{j}\PY{p}{:}\PY{n}{n}\PY{p}{)}\PY{+w}{ }\PY{o}{\PYZhy{}}\PY{+w}{ }\PY{l+m+mi}{2}\PY{o}{*}\PY{n}{u}\PY{o}{*}\PY{p}{(}\PY{n}{u}\PY{o}{\PYZsq{}}\PY{+w}{ }\PY{o}{*}\PY{+w}{ }\PY{n}{A}\PY{p}{(}\PY{n}{j}\PY{p}{:}\PY{n}{m}\PY{p}{,}\PY{+w}{ }\PY{n}{j}\PY{p}{:}\PY{n}{n}\PY{p}{))}
\PY{+w}{    }\PY{k}{end}
\PY{+w}{    }\PY{c+c1}{// A R é a parte triangular superior da A}
\PY{+w}{    }\PY{n}{R}\PY{+w}{ }\PY{p}{=}\PY{+w}{ }\PY{n+nb}{triu}\PY{p}{(}\PY{n}{A}\PY{p}{)}
\PY{k}{endfunction}
\end{Verbatim}
\end{tcolorbox}

    Em seguida, vamos escrever uma versão modificada desse algoritmo, na
qual as iterações sobre as colunas de uma matriz m x n acontecem apenas
até \(m - 1\) colunas se m \(\leq\) n.~A explicação para isso é esta: a
cada iteração, o algoritmo seleciona a coluna seguinte e zera todos os
valores abaixo da diagonal da matriz nessa coluna, ou seja, abaixo do
elemento \(A(j, j)\) da coluna \(j\). No entanto, caso a matriz seja
quadrada ou possua mais colunas que linhas, na \(m\)-ésima iteração,
esse elemento da diagonal será o último elemento daquela coluna, não
havendo nada abaixo dele para ser zerado (para iterações seguintes, ele
apenas não selecionará nada). Portanto, podemos simplesmente
desconsiderar essas últimas iterações.

Feita a explicação, abaixo está a função dessa 2° versão do algoritmo.

    \begin{tcolorbox}[breakable, size=fbox, boxrule=1pt, pad at break*=1mm,colback=cellbackground, colframe=cellborder]
\prompt{In}{incolor}{97}{\boxspacing}
\begin{Verbatim}[commandchars=\\\{\}]
\PY{c+c1}{// Função da versão 2 do algoritmo de Householder para decomposição QR}
\PY{k}{function}\PY{+w}{ }\PY{n+nf}{[U, R] = qr\PYZus{}House\PYZus{}v2}\PY{p}{(}A\PY{p}{)}
\PY{+w}{    }\PY{c+c1}{// Pegando as dimensões da A}
\PY{+w}{    }\PY{p}{[}\PY{n}{m}\PY{p}{,}\PY{+w}{ }\PY{n}{n}\PY{p}{]}\PY{+w}{ }\PY{p}{=}\PY{+w}{ }\PY{n+nb}{size}\PY{p}{(}\PY{n}{A}\PY{p}{)}
\PY{+w}{    }\PY{c+c1}{// Inicializando a matriz U}
\PY{+w}{    }\PY{n}{k}\PY{+w}{ }\PY{p}{=}\PY{+w}{ }\PY{n+nb}{min}\PY{p}{(}\PY{n}{m}\PY{o}{\PYZhy{}}\PY{l+m+mi}{1}\PY{p}{,}\PY{+w}{ }\PY{n}{n}\PY{p}{)}
\PY{+w}{    }\PY{n}{U}\PY{+w}{ }\PY{p}{=}\PY{+w}{ }\PY{n+nb}{zeros}\PY{p}{(}\PY{n}{m}\PY{p}{,}\PY{+w}{ }\PY{n}{k}\PY{p}{)}

\PY{+w}{    }\PY{c+c1}{// Para cada coluna...}
\PY{+w}{    }\PY{k}{for}\PY{+w}{ }\PY{n}{j}\PY{+w}{ }\PY{p}{=}\PY{+w}{ }\PY{l+m+mi}{1}\PY{p}{:}\PY{n}{k}
\PY{+w}{        }\PY{c+c1}{// Pega o vetor que queremos projetar sobre um eixo}
\PY{+w}{        }\PY{n}{x}\PY{+w}{ }\PY{p}{=}\PY{+w}{ }\PY{n}{A}\PY{p}{(}\PY{n}{j}\PY{p}{:}\PY{n}{m}\PY{p}{,}\PY{+w}{ }\PY{n}{j}\PY{p}{)}
\PY{+w}{        }\PY{c+c1}{// Escolhe a melhor das duas projeções possíveis}
\PY{+w}{        }\PY{k}{if}\PY{+w}{ }\PY{n}{x}\PY{p}{(}\PY{l+m+mi}{1}\PY{p}{)}\PY{+w}{ }\PY{o}{\PYZgt{}}\PY{+w}{ }\PY{l+m+mi}{0}\PY{+w}{ }\PY{n+nb}{then}
\PY{+w}{            }\PY{n}{x}\PY{p}{(}\PY{l+m+mi}{1}\PY{p}{)}\PY{+w}{ }\PY{p}{=}\PY{+w}{ }\PY{n}{x}\PY{p}{(}\PY{l+m+mi}{1}\PY{p}{)}\PY{+w}{ }\PY{o}{+}\PY{+w}{ }\PY{n+nb}{norm}\PY{p}{(}\PY{n}{x}\PY{p}{)}
\PY{+w}{        }\PY{k}{else}
\PY{+w}{            }\PY{n}{x}\PY{p}{(}\PY{l+m+mi}{1}\PY{p}{)}\PY{+w}{ }\PY{p}{=}\PY{+w}{ }\PY{n}{x}\PY{p}{(}\PY{l+m+mi}{1}\PY{p}{)}\PY{+w}{ }\PY{o}{\PYZhy{}}\PY{+w}{ }\PY{n+nb}{norm}\PY{p}{(}\PY{n}{x}\PY{p}{)}
\PY{+w}{        }\PY{k}{end}
\PY{+w}{        }\PY{c+c1}{// Calcula u normalizando x}
\PY{+w}{        }\PY{n}{u}\PY{+w}{ }\PY{p}{=}\PY{+w}{ }\PY{n}{x}\PY{o}{/}\PY{n+nb}{norm}\PY{p}{(}\PY{n}{x}\PY{p}{)}
\PY{+w}{        }\PY{c+c1}{// Salva u na matriz U}
\PY{+w}{        }\PY{n}{U}\PY{p}{(}\PY{n}{j}\PY{p}{:}\PY{n}{m}\PY{p}{,}\PY{+w}{ }\PY{n}{j}\PY{p}{)}\PY{+w}{ }\PY{p}{=}\PY{+w}{ }\PY{n}{u}
\PY{+w}{        }\PY{c+c1}{// Triangulariza a parte da matriz correspondente}
\PY{+w}{        }\PY{n}{A}\PY{p}{(}\PY{n}{j}\PY{p}{:}\PY{n}{m}\PY{p}{,}\PY{+w}{ }\PY{n}{j}\PY{p}{:}\PY{n}{n}\PY{p}{)}\PY{+w}{ }\PY{p}{=}\PY{+w}{ }\PY{n}{A}\PY{p}{(}\PY{n}{j}\PY{p}{:}\PY{n}{m}\PY{p}{,}\PY{+w}{ }\PY{n}{j}\PY{p}{:}\PY{n}{n}\PY{p}{)}\PY{+w}{ }\PY{o}{\PYZhy{}}\PY{+w}{ }\PY{l+m+mi}{2}\PY{o}{*}\PY{n}{u}\PY{o}{*}\PY{p}{(}\PY{n}{u}\PY{o}{\PYZsq{}}\PY{+w}{ }\PY{o}{*}\PY{+w}{ }\PY{n}{A}\PY{p}{(}\PY{n}{j}\PY{p}{:}\PY{n}{m}\PY{p}{,}\PY{+w}{ }\PY{n}{j}\PY{p}{:}\PY{n}{n}\PY{p}{))}
\PY{+w}{    }\PY{k}{end}
\PY{+w}{    }\PY{c+c1}{// A R é a parte triangular superior da A}
\PY{+w}{    }\PY{n}{R}\PY{+w}{ }\PY{p}{=}\PY{+w}{ }\PY{n+nb}{triu}\PY{p}{(}\PY{n}{A}\PY{p}{)}
\PY{k}{endfunction}
\end{Verbatim}
\end{tcolorbox}

    Por fim, vamos escrever uma função para construir a matriz \(Q\) dessa
decomposição com base nos vetores presentes nas colunas da matriz \(U\)
fornecida pelo algoritmo.

    \begin{tcolorbox}[breakable, size=fbox, boxrule=1pt, pad at break*=1mm,colback=cellbackground, colframe=cellborder]
\prompt{In}{incolor}{98}{\boxspacing}
\begin{Verbatim}[commandchars=\\\{\}]
\PY{c+c1}{// Função para construir a matriz Q}
\PY{k}{function}\PY{+w}{ }[Q]\PY{+w}{ }\PY{p}{=}\PY{+w}{ }\PY{n+nf}{constroi\PYZus{}Q\PYZus{}house}\PY{p}{(}U\PY{p}{)}
\PY{+w}{    }\PY{c+c1}{// Pegando as dimensões da U}
\PY{+w}{    }\PY{p}{[}\PY{n}{m}\PY{p}{,}\PY{+w}{ }\PY{n}{n}\PY{p}{]}\PY{+w}{ }\PY{p}{=}\PY{+w}{ }\PY{n+nb}{size}\PY{p}{(}\PY{n}{U}\PY{p}{)}

\PY{+w}{    }\PY{c+c1}{// Inicializando Q como a identidade}
\PY{+w}{    }\PY{n}{Q}\PY{+w}{ }\PY{p}{=}\PY{+w}{ }\PY{n+nb}{eye}\PY{p}{(}\PY{n}{m}\PY{p}{,}\PY{+w}{ }\PY{n}{m}\PY{p}{)}

\PY{+w}{    }\PY{c+c1}{// Para cada vetor em U...}
\PY{+w}{    }\PY{k}{for}\PY{+w}{ }\PY{n}{j}\PY{+w}{ }\PY{p}{=}\PY{+w}{ }\PY{l+m+mi}{1}\PY{p}{:}\PY{n}{n}
\PY{+w}{        }\PY{c+c1}{// Pega a coluna da matriz}
\PY{+w}{        }\PY{n}{uj}\PY{+w}{ }\PY{p}{=}\PY{+w}{ }\PY{n}{U}\PY{p}{(:,}\PY{+w}{ }\PY{n}{j}\PY{p}{)}
\PY{+w}{        }\PY{c+c1}{// Calcula a matriz de Householder}
\PY{+w}{        }\PY{n}{Qj}\PY{+w}{ }\PY{p}{=}\PY{+w}{ }\PY{n+nb}{eye}\PY{p}{(}\PY{n}{m}\PY{p}{,}\PY{+w}{ }\PY{n}{m}\PY{p}{)}\PY{+w}{ }\PY{o}{\PYZhy{}}\PY{+w}{ }\PY{l+m+mi}{2}\PY{o}{*}\PY{n}{uj}\PY{o}{*}\PY{n}{uj}\PY{o}{\PYZsq{}}
\PY{+w}{        }\PY{c+c1}{// Multiplica à direita da Q}
\PY{+w}{        }\PY{n}{Q}\PY{+w}{ }\PY{p}{=}\PY{+w}{ }\PY{n}{Q}\PY{+w}{ }\PY{o}{*}\PY{+w}{ }\PY{n}{Qj}
\PY{+w}{    }\PY{k}{end}
\PY{k}{endfunction}
\end{Verbatim}
\end{tcolorbox}

    \hypertarget{item-4.1}{%
\subsection{Item 4.1)}\label{item-4.1}}

Vamos testar a precisão desse método em comparação com os anteriores.

    \begin{tcolorbox}[breakable, size=fbox, boxrule=1pt, pad at break*=1mm,colback=cellbackground, colframe=cellborder]
\prompt{In}{incolor}{99}{\boxspacing}
\begin{Verbatim}[commandchars=\\\{\}]
\PY{c+c1}{// Calculando a U e a R do método de Householder}
\PY{p}{[}\PY{n}{U1HH1}\PY{p}{,}\PY{+w}{ }\PY{n}{R1HH1}\PY{p}{]}\PY{+w}{ }\PY{p}{=}\PY{+w}{ }\PY{n}{qr\PYZus{}House\PYZus{}v1}\PY{p}{(}\PY{n}{A1}\PY{p}{);}
\PY{p}{[}\PY{n}{U1HH2}\PY{p}{,}\PY{+w}{ }\PY{n}{R1HH2}\PY{p}{]}\PY{+w}{ }\PY{p}{=}\PY{+w}{ }\PY{n}{qr\PYZus{}House\PYZus{}v2}\PY{p}{(}\PY{n}{A1}\PY{p}{);}

\PY{c+c1}{// Calculando a matriz Q de Householder}
\PY{n}{Q1HH1}\PY{+w}{ }\PY{p}{=}\PY{+w}{ }\PY{n}{constroi\PYZus{}Q\PYZus{}house}\PY{p}{(}\PY{n}{U1HH1}\PY{p}{);}
\PY{n}{Q1HH2}\PY{+w}{ }\PY{p}{=}\PY{+w}{ }\PY{n}{constroi\PYZus{}Q\PYZus{}house}\PY{p}{(}\PY{n}{U1HH2}\PY{p}{);}

\PY{n+nb}{disp}\PY{p}{(}\PY{l+s}{\PYZdq{}=============== Q*R \PYZhy{} A ================\PYZdq{}}\PY{p}{)}
\PY{n+nb}{disp}\PY{p}{(}\PY{l+s}{\PYZdq{}Gram\PYZhy{}Schmidt\PYZdq{}}\PY{p}{,}\PY{+w}{ }\PY{n+nb}{norm}\PY{p}{(}\PY{n}{Q1}\PY{o}{*}\PY{n}{R1}\PY{+w}{ }\PY{o}{\PYZhy{}}\PY{+w}{ }\PY{n}{A1}\PY{p}{))}
\PY{n+nb}{disp}\PY{p}{(}\PY{l+s}{\PYZdq{}Gram\PYZhy{}Schmidt modificado\PYZdq{}}\PY{p}{,}\PY{+w}{ }\PY{n+nb}{norm}\PY{p}{(}\PY{n}{QM1}\PY{o}{*}\PY{n}{RM1}\PY{+w}{ }\PY{o}{\PYZhy{}}\PY{+w}{ }\PY{n}{A1}\PY{p}{))}
\PY{n+nb}{disp}\PY{p}{(}\PY{l+s}{\PYZdq{}Householder 1\PYZdq{}}\PY{p}{,}\PY{+w}{ }\PY{n+nb}{norm}\PY{p}{(}\PY{n}{Q1HH1}\PY{o}{*}\PY{n}{R1HH1}\PY{+w}{ }\PY{o}{\PYZhy{}}\PY{+w}{ }\PY{n}{A1}\PY{p}{))}
\PY{n+nb}{disp}\PY{p}{(}\PY{l+s}{\PYZdq{}Householder 2\PYZdq{}}\PY{p}{,}\PY{+w}{ }\PY{n+nb}{norm}\PY{p}{(}\PY{n}{Q1HH2}\PY{o}{*}\PY{n}{R1HH2}\PY{+w}{ }\PY{o}{\PYZhy{}}\PY{+w}{ }\PY{n}{A1}\PY{p}{))}
\PY{n+nb}{disp}\PY{p}{(}\PY{l+s}{\PYZdq{}=============== Q\PYZca{}T*Q \PYZhy{} I ===============\PYZdq{}}\PY{p}{)}
\PY{n+nb}{disp}\PY{p}{(}\PY{l+s}{\PYZdq{}Gram\PYZhy{}Schmidt\PYZdq{}}\PY{p}{,}\PY{+w}{ }\PY{n+nb}{norm}\PY{p}{(}\PY{n}{Q1}\PY{o}{\PYZsq{}}\PY{o}{*}\PY{n}{Q1}\PY{+w}{ }\PY{o}{\PYZhy{}}\PY{+w}{ }\PY{n+nb}{eye}\PY{p}{(}\PY{l+m+mi}{2}\PY{p}{,}\PY{+w}{ }\PY{l+m+mi}{2}\PY{p}{)))}
\PY{n+nb}{disp}\PY{p}{(}\PY{l+s}{\PYZdq{}Gram\PYZhy{}Schmidt modificado\PYZdq{}}\PY{p}{,}\PY{+w}{ }\PY{n+nb}{norm}\PY{p}{(}\PY{n}{QM1}\PY{o}{\PYZsq{}}\PY{o}{*}\PY{n}{QM1}\PY{+w}{ }\PY{o}{\PYZhy{}}\PY{+w}{ }\PY{n+nb}{eye}\PY{p}{(}\PY{l+m+mi}{2}\PY{p}{,}\PY{+w}{ }\PY{l+m+mi}{2}\PY{p}{)))}
\PY{n+nb}{disp}\PY{p}{(}\PY{l+s}{\PYZdq{}Householder 1\PYZdq{}}\PY{p}{,}\PY{+w}{ }\PY{n+nb}{norm}\PY{p}{(}\PY{n}{Q1HH1}\PY{o}{\PYZsq{}}\PY{o}{*}\PY{n}{Q1HH1}\PY{+w}{ }\PY{o}{\PYZhy{}}\PY{+w}{ }\PY{n+nb}{eye}\PY{p}{(}\PY{l+m+mi}{3}\PY{p}{,}\PY{+w}{ }\PY{l+m+mi}{3}\PY{p}{)))}
\PY{n+nb}{disp}\PY{p}{(}\PY{l+s}{\PYZdq{}Householder 2\PYZdq{}}\PY{p}{,}\PY{+w}{ }\PY{n+nb}{norm}\PY{p}{(}\PY{n}{Q1HH2}\PY{o}{\PYZsq{}}\PY{o}{*}\PY{n}{Q1HH2}\PY{+w}{ }\PY{o}{\PYZhy{}}\PY{+w}{ }\PY{n+nb}{eye}\PY{p}{(}\PY{l+m+mi}{3}\PY{p}{,}\PY{+w}{ }\PY{l+m+mi}{3}\PY{p}{)))}
\end{Verbatim}
\end{tcolorbox}

    \begin{Verbatim}[commandchars=\\\{\}]
  "=============== Q*R - A ================"

  "Gram-Schmidt"

   1.776D-15

  "Gram-Schmidt modificado"

   1.776D-15

  "Householder 1"

   1.111D-14

  "Householder 2"

   1.111D-14

  "=============== Q\^{}T*Q - I ==============="

  "Gram-Schmidt"

   2.292D-16

  "Gram-Schmidt modificado"

   2.292D-16

  "Householder 1"

   1.882D-15

  "Householder 2"

   1.882D-15
    \end{Verbatim}

    \begin{tcolorbox}[breakable, size=fbox, boxrule=1pt, pad at break*=1mm,colback=cellbackground, colframe=cellborder]
\prompt{In}{incolor}{100}{\boxspacing}
\begin{Verbatim}[commandchars=\\\{\}]
\PY{p}{[}\PY{n}{U2HH1}\PY{p}{,}\PY{+w}{ }\PY{n}{R2HH1}\PY{p}{]}\PY{+w}{ }\PY{p}{=}\PY{+w}{ }\PY{n}{qr\PYZus{}House\PYZus{}v1}\PY{p}{(}\PY{n}{A2}\PY{p}{);}
\PY{p}{[}\PY{n}{U2HH2}\PY{p}{,}\PY{+w}{ }\PY{n}{R2HH2}\PY{p}{]}\PY{+w}{ }\PY{p}{=}\PY{+w}{ }\PY{n}{qr\PYZus{}House\PYZus{}v2}\PY{p}{(}\PY{n}{A2}\PY{p}{);}

\PY{n}{Q2HH1}\PY{+w}{ }\PY{p}{=}\PY{+w}{ }\PY{n}{constroi\PYZus{}Q\PYZus{}house}\PY{p}{(}\PY{n}{U2HH1}\PY{p}{);}
\PY{n}{Q2HH2}\PY{+w}{ }\PY{p}{=}\PY{+w}{ }\PY{n}{constroi\PYZus{}Q\PYZus{}house}\PY{p}{(}\PY{n}{U2HH2}\PY{p}{);}

\PY{n+nb}{disp}\PY{p}{(}\PY{l+s}{\PYZdq{}=============== Q*R \PYZhy{} A ================\PYZdq{}}\PY{p}{)}
\PY{n+nb}{disp}\PY{p}{(}\PY{l+s}{\PYZdq{}Gram\PYZhy{}Schmidt\PYZdq{}}\PY{p}{,}\PY{+w}{ }\PY{n+nb}{norm}\PY{p}{(}\PY{n}{Q2}\PY{o}{*}\PY{n}{R2}\PY{+w}{ }\PY{o}{\PYZhy{}}\PY{+w}{ }\PY{n}{A2}\PY{p}{))}
\PY{n+nb}{disp}\PY{p}{(}\PY{l+s}{\PYZdq{}Gram\PYZhy{}Schmidt modificado\PYZdq{}}\PY{p}{,}\PY{+w}{ }\PY{n+nb}{norm}\PY{p}{(}\PY{n}{QM2}\PY{o}{*}\PY{n}{RM2}\PY{+w}{ }\PY{o}{\PYZhy{}}\PY{+w}{ }\PY{n}{A2}\PY{p}{))}
\PY{n+nb}{disp}\PY{p}{(}\PY{l+s}{\PYZdq{}Householder 1\PYZdq{}}\PY{p}{,}\PY{+w}{ }\PY{n+nb}{norm}\PY{p}{(}\PY{n}{Q2HH1}\PY{o}{*}\PY{n}{R2HH1}\PY{+w}{ }\PY{o}{\PYZhy{}}\PY{+w}{ }\PY{n}{A2}\PY{p}{))}
\PY{n+nb}{disp}\PY{p}{(}\PY{l+s}{\PYZdq{}Householder 2\PYZdq{}}\PY{p}{,}\PY{+w}{ }\PY{n+nb}{norm}\PY{p}{(}\PY{n}{Q2HH2}\PY{o}{*}\PY{n}{R2HH2}\PY{+w}{ }\PY{o}{\PYZhy{}}\PY{+w}{ }\PY{n}{A2}\PY{p}{))}
\PY{n+nb}{disp}\PY{p}{(}\PY{l+s}{\PYZdq{}=============== Q\PYZca{}T*Q \PYZhy{} I ===============\PYZdq{}}\PY{p}{)}
\PY{n+nb}{disp}\PY{p}{(}\PY{l+s}{\PYZdq{}Gram\PYZhy{}Schmidt\PYZdq{}}\PY{p}{,}\PY{+w}{ }\PY{n+nb}{norm}\PY{p}{(}\PY{n}{Q2}\PY{o}{\PYZsq{}}\PY{o}{*}\PY{n}{Q2}\PY{+w}{ }\PY{o}{\PYZhy{}}\PY{+w}{ }\PY{n+nb}{eye}\PY{p}{(}\PY{l+m+mi}{3}\PY{p}{,}\PY{+w}{ }\PY{l+m+mi}{3}\PY{p}{)))}
\PY{n+nb}{disp}\PY{p}{(}\PY{l+s}{\PYZdq{}Gram\PYZhy{}Schmidt modificado\PYZdq{}}\PY{p}{,}\PY{+w}{ }\PY{n+nb}{norm}\PY{p}{(}\PY{n}{QM2}\PY{o}{\PYZsq{}}\PY{o}{*}\PY{n}{QM2}\PY{+w}{ }\PY{o}{\PYZhy{}}\PY{+w}{ }\PY{n+nb}{eye}\PY{p}{(}\PY{l+m+mi}{3}\PY{p}{,}\PY{+w}{ }\PY{l+m+mi}{3}\PY{p}{)))}
\PY{n+nb}{disp}\PY{p}{(}\PY{l+s}{\PYZdq{}Householder 1\PYZdq{}}\PY{p}{,}\PY{+w}{ }\PY{n+nb}{norm}\PY{p}{(}\PY{n}{Q2HH1}\PY{o}{\PYZsq{}}\PY{o}{*}\PY{n}{Q2HH1}\PY{+w}{ }\PY{o}{\PYZhy{}}\PY{+w}{ }\PY{n+nb}{eye}\PY{p}{(}\PY{l+m+mi}{4}\PY{p}{,}\PY{+w}{ }\PY{l+m+mi}{4}\PY{p}{)))}
\PY{n+nb}{disp}\PY{p}{(}\PY{l+s}{\PYZdq{}Householder 2\PYZdq{}}\PY{p}{,}\PY{+w}{ }\PY{n+nb}{norm}\PY{p}{(}\PY{n}{Q2HH2}\PY{o}{\PYZsq{}}\PY{o}{*}\PY{n}{Q2HH2}\PY{+w}{ }\PY{o}{\PYZhy{}}\PY{+w}{ }\PY{n+nb}{eye}\PY{p}{(}\PY{l+m+mi}{4}\PY{p}{,}\PY{+w}{ }\PY{l+m+mi}{4}\PY{p}{)))}
\end{Verbatim}
\end{tcolorbox}

    \begin{Verbatim}[commandchars=\\\{\}]
  "=============== Q*R - A ================"

  "Gram-Schmidt"

   1.891D-15

  "Gram-Schmidt modificado"

   1.891D-15

  "Householder 1"

   2.722D-14

  "Householder 2"

   2.722D-14

  "=============== Q\^{}T*Q - I ==============="

  "Gram-Schmidt"

   2.562D-16

  "Gram-Schmidt modificado"

   2.562D-16

  "Householder 1"

   2.144D-15

  "Householder 2"

   2.144D-15
    \end{Verbatim}

    \begin{tcolorbox}[breakable, size=fbox, boxrule=1pt, pad at break*=1mm,colback=cellbackground, colframe=cellborder]
\prompt{In}{incolor}{101}{\boxspacing}
\begin{Verbatim}[commandchars=\\\{\}]
\PY{p}{[}\PY{n}{U3HH1}\PY{p}{,}\PY{+w}{ }\PY{n}{R3HH1}\PY{p}{]}\PY{+w}{ }\PY{p}{=}\PY{+w}{ }\PY{n}{qr\PYZus{}House\PYZus{}v1}\PY{p}{(}\PY{n}{A3}\PY{p}{);}
\PY{p}{[}\PY{n}{U3HH2}\PY{p}{,}\PY{+w}{ }\PY{n}{R3HH2}\PY{p}{]}\PY{+w}{ }\PY{p}{=}\PY{+w}{ }\PY{n}{qr\PYZus{}House\PYZus{}v2}\PY{p}{(}\PY{n}{A3}\PY{p}{);}

\PY{n}{Q3HH1}\PY{+w}{ }\PY{p}{=}\PY{+w}{ }\PY{n}{constroi\PYZus{}Q\PYZus{}house}\PY{p}{(}\PY{n}{U3HH1}\PY{p}{);}
\PY{n}{Q3HH2}\PY{+w}{ }\PY{p}{=}\PY{+w}{ }\PY{n}{constroi\PYZus{}Q\PYZus{}house}\PY{p}{(}\PY{n}{U3HH2}\PY{p}{);}

\PY{n+nb}{disp}\PY{p}{(}\PY{l+s}{\PYZdq{}=============== Q*R \PYZhy{} A ================\PYZdq{}}\PY{p}{)}
\PY{n+nb}{disp}\PY{p}{(}\PY{l+s}{\PYZdq{}Gram\PYZhy{}Schmidt\PYZdq{}}\PY{p}{,}\PY{+w}{ }\PY{n+nb}{norm}\PY{p}{(}\PY{n}{Q3}\PY{o}{*}\PY{n}{R3}\PY{+w}{ }\PY{o}{\PYZhy{}}\PY{+w}{ }\PY{n}{A3}\PY{p}{))}
\PY{n+nb}{disp}\PY{p}{(}\PY{l+s}{\PYZdq{}Gram\PYZhy{}Schmidt modificado\PYZdq{}}\PY{p}{,}\PY{+w}{ }\PY{n+nb}{norm}\PY{p}{(}\PY{n}{QM3}\PY{o}{*}\PY{n}{RM3}\PY{+w}{ }\PY{o}{\PYZhy{}}\PY{+w}{ }\PY{n}{A3}\PY{p}{))}
\PY{n+nb}{disp}\PY{p}{(}\PY{l+s}{\PYZdq{}Householder 1\PYZdq{}}\PY{p}{,}\PY{+w}{ }\PY{n+nb}{norm}\PY{p}{(}\PY{n}{Q3HH1}\PY{o}{*}\PY{n}{R3HH1}\PY{+w}{ }\PY{o}{\PYZhy{}}\PY{+w}{ }\PY{n}{A3}\PY{p}{))}
\PY{n+nb}{disp}\PY{p}{(}\PY{l+s}{\PYZdq{}Householder 2\PYZdq{}}\PY{p}{,}\PY{+w}{ }\PY{n+nb}{norm}\PY{p}{(}\PY{n}{Q3HH2}\PY{o}{*}\PY{n}{R3HH2}\PY{+w}{ }\PY{o}{\PYZhy{}}\PY{+w}{ }\PY{n}{A3}\PY{p}{))}
\PY{n+nb}{disp}\PY{p}{(}\PY{l+s}{\PYZdq{}=============== Q\PYZca{}T*Q \PYZhy{} I ===============\PYZdq{}}\PY{p}{)}
\PY{n+nb}{disp}\PY{p}{(}\PY{l+s}{\PYZdq{}Gram\PYZhy{}Schmidt\PYZdq{}}\PY{p}{,}\PY{+w}{ }\PY{n+nb}{norm}\PY{p}{(}\PY{n}{Q3}\PY{o}{\PYZsq{}}\PY{o}{*}\PY{n}{Q3}\PY{+w}{ }\PY{o}{\PYZhy{}}\PY{+w}{ }\PY{n+nb}{eye}\PY{p}{(}\PY{l+m+mi}{5}\PY{p}{,}\PY{+w}{ }\PY{l+m+mi}{5}\PY{p}{)))}
\PY{n+nb}{disp}\PY{p}{(}\PY{l+s}{\PYZdq{}Gram\PYZhy{}Schmidt modificado\PYZdq{}}\PY{p}{,}\PY{+w}{ }\PY{n+nb}{norm}\PY{p}{(}\PY{n}{QM3}\PY{o}{\PYZsq{}}\PY{o}{*}\PY{n}{QM3}\PY{+w}{ }\PY{o}{\PYZhy{}}\PY{+w}{ }\PY{n+nb}{eye}\PY{p}{(}\PY{l+m+mi}{5}\PY{p}{,}\PY{+w}{ }\PY{l+m+mi}{5}\PY{p}{)))}
\PY{n+nb}{disp}\PY{p}{(}\PY{l+s}{\PYZdq{}Householder 1\PYZdq{}}\PY{p}{,}\PY{+w}{ }\PY{n+nb}{norm}\PY{p}{(}\PY{n}{Q3HH1}\PY{o}{\PYZsq{}}\PY{o}{*}\PY{n}{Q3HH1}\PY{+w}{ }\PY{o}{\PYZhy{}}\PY{+w}{ }\PY{n+nb}{eye}\PY{p}{(}\PY{l+m+mi}{5}\PY{p}{,}\PY{+w}{ }\PY{l+m+mi}{5}\PY{p}{)))}
\PY{n+nb}{disp}\PY{p}{(}\PY{l+s}{\PYZdq{}Householder 2\PYZdq{}}\PY{p}{,}\PY{+w}{ }\PY{n+nb}{norm}\PY{p}{(}\PY{n}{Q3HH2}\PY{o}{\PYZsq{}}\PY{o}{*}\PY{n}{Q3HH2}\PY{+w}{ }\PY{o}{\PYZhy{}}\PY{+w}{ }\PY{n+nb}{eye}\PY{p}{(}\PY{l+m+mi}{5}\PY{p}{,}\PY{+w}{ }\PY{l+m+mi}{5}\PY{p}{)))}
\end{Verbatim}
\end{tcolorbox}

    \begin{Verbatim}[commandchars=\\\{\}]
  "=============== Q*R - A ================"

  "Gram-Schmidt"

   8.882D-16

  "Gram-Schmidt modificado"

   1.776D-15

  "Householder 1"

   1.308D-14

  "Householder 2"

   1.308D-14

  "=============== Q\^{}T*Q - I ==============="

  "Gram-Schmidt"

   6.279D-16

  "Gram-Schmidt modificado"

   5.514D-16

  "Householder 1"

   1.404D-15

  "Householder 2"

   1.404D-15
    \end{Verbatim}

    Assim, observa-se que o método de decomposição \(QR\) de Householder tem
uma precisão bem semelhante aos métodos de Gram-Schmidt.

    \hypertarget{item-4.2}{%
\subsection{Item 4.2)}\label{item-4.2}}

\hypertarget{matriz-muxe1gica-7x7}{%
\subsubsection{1. Matriz mágica 7x7}\label{matriz-muxe1gica-7x7}}

A seguir, testaremos nossas funções com algumas matrizes especiais,
geralmente instáveis. Começaremos com uma matriz mágica 7x7, na qual as
somas dos elementos de cada linha, coluna e diagonal é a mesma.

    \begin{tcolorbox}[breakable, size=fbox, boxrule=1pt, pad at break*=1mm,colback=cellbackground, colframe=cellborder]
\prompt{In}{incolor}{102}{\boxspacing}
\begin{Verbatim}[commandchars=\\\{\}]
\PY{n}{M1}\PY{+w}{ }\PY{p}{=}\PY{+w}{ }\PY{n+nb}{testmatrix}\PY{p}{(}\PY{l+s}{\PYZsq{}}\PY{l+s}{magi\PYZsq{}}\PY{p}{,}\PY{+w}{ }\PY{l+m+mi}{7}\PY{p}{)}
\end{Verbatim}
\end{tcolorbox}

    \begin{Verbatim}[commandchars=\\\{\}]
 M1  =
   30.   39.   48.   1.    10.   19.   28.
   38.   47.   7.    9.    18.   27.   29.
   46.   6.    8.    17.   26.   35.   37.
   5.    14.   16.   25.   34.   36.   45.
   13.   15.   24.   33.   42.   44.   4.
   21.   23.   32.   41.   43.   3.    12.
   22.   31.   40.   49.   2.    11.   20.
    \end{Verbatim}

    \begin{tcolorbox}[breakable, size=fbox, boxrule=1pt, pad at break*=1mm,colback=cellbackground, colframe=cellborder]
\prompt{In}{incolor}{103}{\boxspacing}
\begin{Verbatim}[commandchars=\\\{\}]
\PY{c+c1}{// Calculando a decomposição QR pelos métodos de Gram\PYZhy{}Schmidt}
\PY{p}{[}\PY{n}{QM1GS}\PY{p}{,}\PY{+w}{ }\PY{n}{RM1GS}\PY{p}{]}\PY{+w}{ }\PY{p}{=}\PY{+w}{ }\PY{n}{qr\PYZus{}GS}\PY{p}{(}\PY{n}{M1}\PY{p}{);}
\PY{p}{[}\PY{n}{QM1GSM}\PY{p}{,}\PY{+w}{ }\PY{n}{RM1GSM}\PY{p}{]}\PY{+w}{ }\PY{p}{=}\PY{+w}{ }\PY{n}{qr\PYZus{}GSM}\PY{p}{(}\PY{n}{M1}\PY{p}{);}

\PY{c+c1}{// Calculando a decomposição QR pelo método de Householder}
\PY{p}{[}\PY{n}{UM1HH}\PY{p}{,}\PY{+w}{ }\PY{n}{RM1HH}\PY{p}{]}\PY{+w}{ }\PY{p}{=}\PY{+w}{ }\PY{n}{qr\PYZus{}House\PYZus{}v2}\PY{p}{(}\PY{n}{M1}\PY{p}{);}
\PY{n}{QM1HH}\PY{+w}{ }\PY{p}{=}\PY{+w}{ }\PY{n}{constroi\PYZus{}Q\PYZus{}house}\PY{p}{(}\PY{n}{UM1HH}\PY{p}{);}

\PY{c+c1}{// Calculando a decomposição QR com a função do Scilab}
\PY{p}{[}\PY{n}{QM1Sci}\PY{p}{,}\PY{+w}{ }\PY{n}{RM1Sci}\PY{p}{]}\PY{+w}{ }\PY{p}{=}\PY{+w}{ }\PY{n+nb}{qr}\PY{p}{(}\PY{n}{M1}\PY{p}{);}

\PY{n+nb}{disp}\PY{p}{(}\PY{l+s}{\PYZdq{}=============== Q*R \PYZhy{} A ================\PYZdq{}}\PY{p}{)}
\PY{n+nb}{disp}\PY{p}{(}\PY{l+s}{\PYZdq{}Gram\PYZhy{}Schmidt\PYZdq{}}\PY{p}{,}\PY{+w}{ }\PY{n+nb}{norm}\PY{p}{(}\PY{n}{QM1GS}\PY{o}{*}\PY{n}{RM1GS}\PY{+w}{ }\PY{o}{\PYZhy{}}\PY{+w}{ }\PY{n}{M1}\PY{p}{))}
\PY{n+nb}{disp}\PY{p}{(}\PY{l+s}{\PYZdq{}Gram\PYZhy{}Schmidt modificado\PYZdq{}}\PY{p}{,}\PY{+w}{ }\PY{n+nb}{norm}\PY{p}{(}\PY{n}{QM1GSM}\PY{o}{*}\PY{n}{RM1GSM}\PY{+w}{ }\PY{o}{\PYZhy{}}\PY{+w}{ }\PY{n}{M1}\PY{p}{))}
\PY{n+nb}{disp}\PY{p}{(}\PY{l+s}{\PYZdq{}Householder\PYZdq{}}\PY{p}{,}\PY{+w}{ }\PY{n+nb}{norm}\PY{p}{(}\PY{n}{QM1HH}\PY{o}{*}\PY{n}{RM1HH}\PY{+w}{ }\PY{o}{\PYZhy{}}\PY{+w}{ }\PY{n}{M1}\PY{p}{))}
\PY{n+nb}{disp}\PY{p}{(}\PY{l+s}{\PYZdq{}Scilab\PYZdq{}}\PY{p}{,}\PY{+w}{ }\PY{n+nb}{norm}\PY{p}{(}\PY{n}{QM1Sci}\PY{o}{*}\PY{n}{RM1Sci}\PY{+w}{ }\PY{o}{\PYZhy{}}\PY{+w}{ }\PY{n}{M1}\PY{p}{))}
\PY{n+nb}{disp}\PY{p}{(}\PY{l+s}{\PYZdq{}=============== Q\PYZca{}T*Q \PYZhy{} I ===============\PYZdq{}}\PY{p}{)}
\PY{n+nb}{disp}\PY{p}{(}\PY{l+s}{\PYZdq{}Gram\PYZhy{}Schmidt\PYZdq{}}\PY{p}{,}\PY{+w}{ }\PY{n+nb}{norm}\PY{p}{(}\PY{n}{QM1GS}\PY{o}{\PYZsq{}}\PY{o}{*}\PY{n}{QM1GS}\PY{+w}{ }\PY{o}{\PYZhy{}}\PY{+w}{ }\PY{n+nb}{eye}\PY{p}{(}\PY{l+m+mi}{7}\PY{p}{,}\PY{+w}{ }\PY{l+m+mi}{7}\PY{p}{)))}
\PY{n+nb}{disp}\PY{p}{(}\PY{l+s}{\PYZdq{}Gram\PYZhy{}Schmidt modificado\PYZdq{}}\PY{p}{,}\PY{+w}{ }\PY{n+nb}{norm}\PY{p}{(}\PY{n}{QM1GSM}\PY{o}{\PYZsq{}}\PY{o}{*}\PY{n}{QM1GSM}\PY{+w}{ }\PY{o}{\PYZhy{}}\PY{+w}{ }\PY{n+nb}{eye}\PY{p}{(}\PY{l+m+mi}{7}\PY{p}{,}\PY{+w}{ }\PY{l+m+mi}{7}\PY{p}{)))}
\PY{n+nb}{disp}\PY{p}{(}\PY{l+s}{\PYZdq{}Householder\PYZdq{}}\PY{p}{,}\PY{+w}{ }\PY{n+nb}{norm}\PY{p}{(}\PY{n}{QM1HH}\PY{o}{\PYZsq{}}\PY{o}{*}\PY{n}{QM1HH}\PY{+w}{ }\PY{o}{\PYZhy{}}\PY{+w}{ }\PY{n+nb}{eye}\PY{p}{(}\PY{l+m+mi}{7}\PY{p}{,}\PY{+w}{ }\PY{l+m+mi}{7}\PY{p}{)))}
\PY{n+nb}{disp}\PY{p}{(}\PY{l+s}{\PYZdq{}Scilab\PYZdq{}}\PY{p}{,}\PY{+w}{ }\PY{n+nb}{norm}\PY{p}{(}\PY{n}{QM1Sci}\PY{o}{\PYZsq{}}\PY{o}{*}\PY{n}{QM1Sci}\PY{+w}{ }\PY{o}{\PYZhy{}}\PY{+w}{ }\PY{n+nb}{eye}\PY{p}{(}\PY{l+m+mi}{7}\PY{p}{,}\PY{+w}{ }\PY{l+m+mi}{7}\PY{p}{)))}
\end{Verbatim}
\end{tcolorbox}

    \begin{Verbatim}[commandchars=\\\{\}]
  "=============== Q*R - A ================"

  "Gram-Schmidt"

   7.105D-15

  "Gram-Schmidt modificado"

   1.029D-14

  "Householder"

   5.506D-14

  "Scilab"

   9.110D-14

  "=============== Q\^{}T*Q - I ==============="

  "Gram-Schmidt"

   1.574D-15

  "Gram-Schmidt modificado"

   1.017D-15

  "Householder"

   6.063D-16

  "Scilab"

   6.171D-16
    \end{Verbatim}

    Parece que todos os métodos obtiveram resultados bem precisos para
ortogonalizar essa matriz mágica.

\hypertarget{matriz-de-hilbert-7x7}{%
\subsubsection{2. Matriz de Hilbert 7x7}\label{matriz-de-hilbert-7x7}}

Agora, vamos testar com a inversa de uma matriz de Hilbert, cuja
definição é a seguinte para cada elemento \((i, j)\):

\[
H_{i, j} = \dfrac{1}{i+j-1}
\]

    \begin{tcolorbox}[breakable, size=fbox, boxrule=1pt, pad at break*=1mm,colback=cellbackground, colframe=cellborder]
\prompt{In}{incolor}{104}{\boxspacing}
\begin{Verbatim}[commandchars=\\\{\}]
\PY{n}{H}\PY{+w}{ }\PY{p}{=}\PY{+w}{ }\PY{n+nb}{testmatrix}\PY{p}{(}\PY{l+s}{\PYZsq{}}\PY{l+s}{hilb\PYZsq{}}\PY{p}{,}\PY{+w}{ }\PY{l+m+mi}{7}\PY{p}{)}
\end{Verbatim}
\end{tcolorbox}

    \begin{Verbatim}[commandchars=\\\{\}]
 H  =
   49.     -1176.      8820.      -29400.      48510.     -38808.      12012.
  -1176.    37632.    -317520.     1128960.   -1940400.    1596672.   -504504.
   8820.   -317520.    2857680.   -10584000.   18711000.  -15717240.   5045040.
  -29400.   1128960.  -10584000.   40320000.  -72765000.   62092800.  -20180160.
   48510.  -1940400.   18711000.  -72765000.   1.334D+08  -1.153D+08   37837800.
  -38808.   1596672.  -15717240.   62092800.  -1.153D+08   1.006D+08  -33297264.
   12012.  -504504.    5045040.   -20180160.   37837800.  -33297264.   11099088.
    \end{Verbatim}

    \begin{tcolorbox}[breakable, size=fbox, boxrule=1pt, pad at break*=1mm,colback=cellbackground, colframe=cellborder]
\prompt{In}{incolor}{105}{\boxspacing}
\begin{Verbatim}[commandchars=\\\{\}]
\PY{c+c1}{// Calculando a decomposição QR pelos métodos de Gram\PYZhy{}Schmidt}
\PY{p}{[}\PY{n}{QHGS}\PY{p}{,}\PY{+w}{ }\PY{n}{RHGS}\PY{p}{]}\PY{+w}{ }\PY{p}{=}\PY{+w}{ }\PY{n}{qr\PYZus{}GS}\PY{p}{(}\PY{n}{H}\PY{p}{);}
\PY{p}{[}\PY{n}{QHGSM}\PY{p}{,}\PY{+w}{ }\PY{n}{RHGSM}\PY{p}{]}\PY{+w}{ }\PY{p}{=}\PY{+w}{ }\PY{n}{qr\PYZus{}GSM}\PY{p}{(}\PY{n}{H}\PY{p}{);}

\PY{c+c1}{// Calculando a decomposição QR pelo método de Householder}
\PY{p}{[}\PY{n}{UHHH}\PY{p}{,}\PY{+w}{ }\PY{n}{RHHH}\PY{p}{]}\PY{+w}{ }\PY{p}{=}\PY{+w}{ }\PY{n}{qr\PYZus{}House\PYZus{}v2}\PY{p}{(}\PY{n}{H}\PY{p}{);}
\PY{n}{QHHH}\PY{+w}{ }\PY{p}{=}\PY{+w}{ }\PY{n}{constroi\PYZus{}Q\PYZus{}house}\PY{p}{(}\PY{n}{UHHH}\PY{p}{);}

\PY{c+c1}{// Calculando a decomposição QR com a função do Scilab}
\PY{p}{[}\PY{n}{QHSci}\PY{p}{,}\PY{+w}{ }\PY{n}{RHSci}\PY{p}{]}\PY{+w}{ }\PY{p}{=}\PY{+w}{ }\PY{n+nb}{qr}\PY{p}{(}\PY{n}{H}\PY{p}{);}

\PY{n+nb}{disp}\PY{p}{(}\PY{l+s}{\PYZdq{}=============== Q*R \PYZhy{} A ================\PYZdq{}}\PY{p}{)}
\PY{n+nb}{disp}\PY{p}{(}\PY{l+s}{\PYZdq{}Gram\PYZhy{}Schmidt\PYZdq{}}\PY{p}{,}\PY{+w}{ }\PY{n+nb}{norm}\PY{p}{(}\PY{n}{QHGS}\PY{o}{*}\PY{n}{RHGS}\PY{+w}{ }\PY{o}{\PYZhy{}}\PY{+w}{ }\PY{n}{H}\PY{p}{))}
\PY{n+nb}{disp}\PY{p}{(}\PY{l+s}{\PYZdq{}Gram\PYZhy{}Schmidt modificado\PYZdq{}}\PY{p}{,}\PY{+w}{ }\PY{n+nb}{norm}\PY{p}{(}\PY{n}{QHGSM}\PY{o}{*}\PY{n}{RHGSM}\PY{+w}{ }\PY{o}{\PYZhy{}}\PY{+w}{ }\PY{n}{H}\PY{p}{))}
\PY{n+nb}{disp}\PY{p}{(}\PY{l+s}{\PYZdq{}Householder\PYZdq{}}\PY{p}{,}\PY{+w}{ }\PY{n+nb}{norm}\PY{p}{(}\PY{n}{QHHH}\PY{o}{*}\PY{n}{RHHH}\PY{+w}{ }\PY{o}{\PYZhy{}}\PY{+w}{ }\PY{n}{H}\PY{p}{))}
\PY{n+nb}{disp}\PY{p}{(}\PY{l+s}{\PYZdq{}Scilab\PYZdq{}}\PY{p}{,}\PY{+w}{ }\PY{n+nb}{norm}\PY{p}{(}\PY{n}{QHSci}\PY{o}{*}\PY{n}{RHSci}\PY{+w}{ }\PY{o}{\PYZhy{}}\PY{+w}{ }\PY{n}{H}\PY{p}{))}
\PY{n+nb}{disp}\PY{p}{(}\PY{l+s}{\PYZdq{}=============== Q\PYZca{}T*Q \PYZhy{} I ===============\PYZdq{}}\PY{p}{)}
\PY{n+nb}{disp}\PY{p}{(}\PY{l+s}{\PYZdq{}Gram\PYZhy{}Schmidt\PYZdq{}}\PY{p}{,}\PY{+w}{ }\PY{n+nb}{norm}\PY{p}{(}\PY{n}{QHGS}\PY{o}{\PYZsq{}}\PY{o}{*}\PY{n}{QHGS}\PY{+w}{ }\PY{o}{\PYZhy{}}\PY{+w}{ }\PY{n+nb}{eye}\PY{p}{(}\PY{l+m+mi}{7}\PY{p}{,}\PY{+w}{ }\PY{l+m+mi}{7}\PY{p}{)))}
\PY{n+nb}{disp}\PY{p}{(}\PY{l+s}{\PYZdq{}Gram\PYZhy{}Schmidt modificado\PYZdq{}}\PY{p}{,}\PY{+w}{ }\PY{n+nb}{norm}\PY{p}{(}\PY{n}{QHGSM}\PY{o}{\PYZsq{}}\PY{o}{*}\PY{n}{QHGSM}\PY{+w}{ }\PY{o}{\PYZhy{}}\PY{+w}{ }\PY{n+nb}{eye}\PY{p}{(}\PY{l+m+mi}{7}\PY{p}{,}\PY{+w}{ }\PY{l+m+mi}{7}\PY{p}{)))}
\PY{n+nb}{disp}\PY{p}{(}\PY{l+s}{\PYZdq{}Householder\PYZdq{}}\PY{p}{,}\PY{+w}{ }\PY{n+nb}{norm}\PY{p}{(}\PY{n}{QHHH}\PY{o}{\PYZsq{}}\PY{o}{*}\PY{n}{QHHH}\PY{+w}{ }\PY{o}{\PYZhy{}}\PY{+w}{ }\PY{n+nb}{eye}\PY{p}{(}\PY{l+m+mi}{7}\PY{p}{,}\PY{+w}{ }\PY{l+m+mi}{7}\PY{p}{)))}
\PY{n+nb}{disp}\PY{p}{(}\PY{l+s}{\PYZdq{}Scilab\PYZdq{}}\PY{p}{,}\PY{+w}{ }\PY{n+nb}{norm}\PY{p}{(}\PY{n}{QHSci}\PY{o}{\PYZsq{}}\PY{o}{*}\PY{n}{QHSci}\PY{+w}{ }\PY{o}{\PYZhy{}}\PY{+w}{ }\PY{n+nb}{eye}\PY{p}{(}\PY{l+m+mi}{7}\PY{p}{,}\PY{+w}{ }\PY{l+m+mi}{7}\PY{p}{)))}
\end{Verbatim}
\end{tcolorbox}

    \begin{Verbatim}[commandchars=\\\{\}]
  "=============== Q*R - A ================"

  "Gram-Schmidt"

   0.

  "Gram-Schmidt modificado"

   8.975D-09

  "Householder"

   0.0000001

  "Scilab"

   3.688D-08

  "=============== Q\^{}T*Q - I ==============="

  "Gram-Schmidt"

   0.9852109

  "Gram-Schmidt modificado"

   1.933D-09

  "Householder"

   1.447D-15

  "Scilab"

   3.746D-16
    \end{Verbatim}

    Curiosamente, o método de Gram-Schmidt normal obteve precisão máxima no
quesito da decomposição da matriz \(H\), mas ficou com um erro muito
grande na ortogonalidade da matriz \(Q\). Isso provavelmente ocorre
devido a sua abordagem direta e simples na decomposição da matriz e ao
fato de ela não preservar exatamente a ortogonalidade entre as colunas
de \(Q\), especialmente em matrizes mal-condicionadas como essa.

Com relação ao método modificado, seus resultados ficaram com uma
precisão razoável, melhor que a do método normal. Por ser uma melhoria,
o algoritmo atualizado geralmente mantém a precisão da decomposição da
matriz e visa melhorar a ortogonalidade de \(Q\).

Olhando para o Householder, ele obteve o resultado inverso do
Gram-Schmidt: sua precisão na decomposição da \(H\) não foi tão boa, mas
a ortogonalidade da \(Q\) obtida ficou muito alta. Uma possível razão
para isso é que esse método preserva a ortogonalidade exata entre as
colunas de \(Q\), mesmo com a decomposição não sendo tão precisa.

Por fim, a função do Scilab ficou com resultados bons nos dois quesitos.

\hypertarget{matriz-muxe1gica-6x6}{%
\subsubsection{3. Matriz mágica 6x6}\label{matriz-muxe1gica-6x6}}

Por fim, vamos testar com outra matriz mágica, mas agora 6x6.

    \begin{tcolorbox}[breakable, size=fbox, boxrule=1pt, pad at break*=1mm,colback=cellbackground, colframe=cellborder]
\prompt{In}{incolor}{106}{\boxspacing}
\begin{Verbatim}[commandchars=\\\{\}]
\PY{n}{M2}\PY{+w}{ }\PY{p}{=}\PY{+w}{ }\PY{n+nb}{testmatrix}\PY{p}{(}\PY{l+s}{\PYZsq{}}\PY{l+s}{magi\PYZsq{}}\PY{p}{,}\PY{+w}{ }\PY{l+m+mi}{6}\PY{p}{)}
\end{Verbatim}
\end{tcolorbox}

    \begin{Verbatim}[commandchars=\\\{\}]
 M2  =
   35.   1.    6.    26.   19.   24.
   3.    32.   7.    21.   23.   25.
   31.   9.    2.    22.   27.   20.
   8.    28.   33.   17.   10.   15.
   30.   5.    34.   12.   14.   16.
   4.    36.   29.   13.   18.   11.
    \end{Verbatim}

    \begin{tcolorbox}[breakable, size=fbox, boxrule=1pt, pad at break*=1mm,colback=cellbackground, colframe=cellborder]
\prompt{In}{incolor}{107}{\boxspacing}
\begin{Verbatim}[commandchars=\\\{\}]
\PY{c+c1}{// Calculando a decomposição QR pelos métodos de Gram\PYZhy{}Schmidt}
\PY{p}{[}\PY{n}{QM2GS}\PY{p}{,}\PY{+w}{ }\PY{n}{RM2GS}\PY{p}{]}\PY{+w}{ }\PY{p}{=}\PY{+w}{ }\PY{n}{qr\PYZus{}GS}\PY{p}{(}\PY{n}{M2}\PY{p}{);}
\PY{p}{[}\PY{n}{QM2GSM}\PY{p}{,}\PY{+w}{ }\PY{n}{RM2GSM}\PY{p}{]}\PY{+w}{ }\PY{p}{=}\PY{+w}{ }\PY{n}{qr\PYZus{}GSM}\PY{p}{(}\PY{n}{M2}\PY{p}{);}

\PY{c+c1}{// Calculando a decomposição QR pelo método de Householder}
\PY{p}{[}\PY{n}{UM2HH}\PY{p}{,}\PY{+w}{ }\PY{n}{RM2HH}\PY{p}{]}\PY{+w}{ }\PY{p}{=}\PY{+w}{ }\PY{n}{qr\PYZus{}House\PYZus{}v2}\PY{p}{(}\PY{n}{M2}\PY{p}{);}
\PY{n}{QM2HH}\PY{+w}{ }\PY{p}{=}\PY{+w}{ }\PY{n}{constroi\PYZus{}Q\PYZus{}house}\PY{p}{(}\PY{n}{UM2HH}\PY{p}{);}

\PY{c+c1}{// Calculando a decomposição QR com a função do Scilab}
\PY{p}{[}\PY{n}{QM2Sci}\PY{p}{,}\PY{+w}{ }\PY{n}{RM2Sci}\PY{p}{]}\PY{+w}{ }\PY{p}{=}\PY{+w}{ }\PY{n+nb}{qr}\PY{p}{(}\PY{n}{M2}\PY{p}{);}

\PY{n+nb}{disp}\PY{p}{(}\PY{l+s}{\PYZdq{}=============== Q*R \PYZhy{} A ================\PYZdq{}}\PY{p}{)}
\PY{n+nb}{disp}\PY{p}{(}\PY{l+s}{\PYZdq{}Gram\PYZhy{}Schmidt\PYZdq{}}\PY{p}{,}\PY{+w}{ }\PY{n+nb}{norm}\PY{p}{(}\PY{n}{QM2GS}\PY{o}{*}\PY{n}{RM2GS}\PY{+w}{ }\PY{o}{\PYZhy{}}\PY{+w}{ }\PY{n}{M2}\PY{p}{))}
\PY{n+nb}{disp}\PY{p}{(}\PY{l+s}{\PYZdq{}Gram\PYZhy{}Schmidt modificado\PYZdq{}}\PY{p}{,}\PY{+w}{ }\PY{n+nb}{norm}\PY{p}{(}\PY{n}{QM2GSM}\PY{o}{*}\PY{n}{RM2GSM}\PY{+w}{ }\PY{o}{\PYZhy{}}\PY{+w}{ }\PY{n}{M2}\PY{p}{))}
\PY{n+nb}{disp}\PY{p}{(}\PY{l+s}{\PYZdq{}Householder\PYZdq{}}\PY{p}{,}\PY{+w}{ }\PY{n+nb}{norm}\PY{p}{(}\PY{n}{QM2HH}\PY{o}{*}\PY{n}{RM2HH}\PY{+w}{ }\PY{o}{\PYZhy{}}\PY{+w}{ }\PY{n}{M2}\PY{p}{))}
\PY{n+nb}{disp}\PY{p}{(}\PY{l+s}{\PYZdq{}Scilab\PYZdq{}}\PY{p}{,}\PY{+w}{ }\PY{n+nb}{norm}\PY{p}{(}\PY{n}{QM2Sci}\PY{o}{*}\PY{n}{RM2Sci}\PY{+w}{ }\PY{o}{\PYZhy{}}\PY{+w}{ }\PY{n}{M2}\PY{p}{))}
\PY{n+nb}{disp}\PY{p}{(}\PY{l+s}{\PYZdq{}=============== Q\PYZca{}T*Q \PYZhy{} I ===============\PYZdq{}}\PY{p}{)}
\PY{n+nb}{disp}\PY{p}{(}\PY{l+s}{\PYZdq{}Gram\PYZhy{}Schmidt\PYZdq{}}\PY{p}{,}\PY{+w}{ }\PY{n+nb}{norm}\PY{p}{(}\PY{n}{QM2GS}\PY{o}{\PYZsq{}}\PY{o}{*}\PY{n}{QM2GS}\PY{+w}{ }\PY{o}{\PYZhy{}}\PY{+w}{ }\PY{n+nb}{eye}\PY{p}{(}\PY{l+m+mi}{6}\PY{p}{,}\PY{+w}{ }\PY{l+m+mi}{6}\PY{p}{)))}
\PY{n+nb}{disp}\PY{p}{(}\PY{l+s}{\PYZdq{}Gram\PYZhy{}Schmidt modificado\PYZdq{}}\PY{p}{,}\PY{+w}{ }\PY{n+nb}{norm}\PY{p}{(}\PY{n}{QM2GSM}\PY{o}{\PYZsq{}}\PY{o}{*}\PY{n}{QM2GSM}\PY{+w}{ }\PY{o}{\PYZhy{}}\PY{+w}{ }\PY{n+nb}{eye}\PY{p}{(}\PY{l+m+mi}{6}\PY{p}{,}\PY{+w}{ }\PY{l+m+mi}{6}\PY{p}{)))}
\PY{n+nb}{disp}\PY{p}{(}\PY{l+s}{\PYZdq{}Householder\PYZdq{}}\PY{p}{,}\PY{+w}{ }\PY{n+nb}{norm}\PY{p}{(}\PY{n}{QM2HH}\PY{o}{\PYZsq{}}\PY{o}{*}\PY{n}{QM2HH}\PY{+w}{ }\PY{o}{\PYZhy{}}\PY{+w}{ }\PY{n+nb}{eye}\PY{p}{(}\PY{l+m+mi}{6}\PY{p}{,}\PY{+w}{ }\PY{l+m+mi}{6}\PY{p}{)))}
\PY{n+nb}{disp}\PY{p}{(}\PY{l+s}{\PYZdq{}Scilab\PYZdq{}}\PY{p}{,}\PY{+w}{ }\PY{n+nb}{norm}\PY{p}{(}\PY{n}{QM2Sci}\PY{o}{\PYZsq{}}\PY{o}{*}\PY{n}{QM2Sci}\PY{+w}{ }\PY{o}{\PYZhy{}}\PY{+w}{ }\PY{n+nb}{eye}\PY{p}{(}\PY{l+m+mi}{6}\PY{p}{,}\PY{+w}{ }\PY{l+m+mi}{6}\PY{p}{)))}
\end{Verbatim}
\end{tcolorbox}

    \begin{Verbatim}[commandchars=\\\{\}]
  "=============== Q*R - A ================"

  "Gram-Schmidt"

   7.957D-15

  "Gram-Schmidt modificado"

   7.960D-15

  "Householder"

   1.750D-14

  "Scilab"

   4.101D-14

  "=============== Q\^{}T*Q - I ==============="

  "Gram-Schmidt"

   0.9965096

  "Gram-Schmidt modificado"

   0.9374063

  "Householder"

   1.166D-15

  "Scilab"

   6.629D-16
    \end{Verbatim}

    Diferentemente da matriz mágica 7x7 testada anteriormente, essa 6x6
trouxe alguns problemas para nossos algoritmos. Isso se deve à peculiar
característica dessa classe de matrizes de que seus espécimes de ordem
ímpar são bem-condicionados, enquanto que os de ordem par são
mal-condicionados.

Todos os métodos obtiveram aproximações boas na decomposição da matriz
em \(Q\) e \(R\).

No entanto, o Gram-Schmidt não obteve sucesso na ortogonalidade da
\(Q\), com ambas as versões ficando com erros altos. Uma possível causa
para isso é a falta de robustez do método de Gram-Schmidt para garantir
a ortogonalidade exata entre as colunas de \(Q\) em comparação com os
outros métodos, especialmente com matrizes mal-condicionadas,
introduzindo erros numéricos que afetam essa ortogonalidade.

Por outro lado, tanto o método de Householder quanto o do Scilab
conseguiram resultados muito bons nesse quesito.

\hypertarget{exercuxedcio-5}{%
\section{Exercício 5}\label{exercuxedcio-5}}

Por último, vamos usar a decomposição \(QR\) para encontrar os
autovalores de uma matriz simétrica. A função para isso está a seguir.

    \begin{tcolorbox}[breakable, size=fbox, boxrule=1pt, pad at break*=1mm,colback=cellbackground, colframe=cellborder]
\prompt{In}{incolor}{108}{\boxspacing}
\begin{Verbatim}[commandchars=\\\{\}]
\PY{c+c1}{// Função para calcular os autovalores de uma matriz por meio da decomposição QR}
\PY{k}{function}\PY{+w}{ }[S]\PY{+w}{ }\PY{p}{=}\PY{+w}{ }\PY{n+nf}{espectro}\PY{p}{(}A, tol\PY{p}{)}
\PY{+w}{    }\PY{c+c1}{// Inicializando os vetores atual e anterior distantes para garantir a entrada no loop}
\PY{+w}{    }\PY{n}{S}\PY{+w}{ }\PY{p}{=}\PY{+w}{ }\PY{p}{[}\PY{n+no}{\PYZpc{}inf}\PY{p}{]}
\PY{+w}{    }\PY{n}{S0}\PY{+w}{ }\PY{p}{=}\PY{+w}{ }\PY{p}{[}\PY{l+m+mi}{0}\PY{p}{]}

\PY{+w}{    }\PY{c+c1}{// Enquanto o vetor não convergir...}
\PY{+w}{    }\PY{k}{while}\PY{+w}{ }\PY{n+nb}{norm}\PY{p}{(}\PY{n}{S}\PY{+w}{ }\PY{o}{\PYZhy{}}\PY{+w}{ }\PY{n}{S0}\PY{p}{,}\PY{+w}{ }\PY{l+s}{\PYZsq{}}\PY{l+s}{inf\PYZsq{}}\PY{p}{)}\PY{+w}{ }\PY{o}{\PYZgt{}}\PY{+w}{ }\PY{n}{tol}
\PY{+w}{        }\PY{c+c1}{// Salva o valor anterior}
\PY{+w}{        }\PY{n}{S0}\PY{+w}{ }\PY{p}{=}\PY{+w}{ }\PY{n}{S}
\PY{+w}{        }\PY{c+c1}{// Calcula a decomposição QR da matriz}
\PY{+w}{        }\PY{p}{[}\PY{n}{Q}\PY{p}{,}\PY{+w}{ }\PY{n}{R}\PY{p}{]}\PY{+w}{ }\PY{p}{=}\PY{+w}{ }\PY{n}{qr\PYZus{}GSM}\PY{p}{(}\PY{n}{A}\PY{p}{)}
\PY{+w}{        }\PY{c+c1}{// Substitui ela por RQ}
\PY{+w}{        }\PY{n}{A}\PY{+w}{ }\PY{p}{=}\PY{+w}{ }\PY{n}{R}\PY{o}{*}\PY{n}{Q}
\PY{+w}{        }\PY{c+c1}{// Pega a diagonal dessa matriz como os autovalores}
\PY{+w}{        }\PY{n}{S}\PY{+w}{ }\PY{p}{=}\PY{+w}{ }\PY{n+nb}{diag}\PY{p}{(}\PY{n}{A}\PY{p}{)}
\PY{+w}{    }\PY{k}{end}
\PY{k}{endfunction}
\end{Verbatim}
\end{tcolorbox}

    \begin{Verbatim}[commandchars=\\\{\}]
Warning : redefining function: espectro                . Use funcprot(0) to
avoid this message
    \end{Verbatim}

    Vamos testar com algumas matrizes.

    \begin{tcolorbox}[breakable, size=fbox, boxrule=1pt, pad at break*=1mm,colback=cellbackground, colframe=cellborder]
\prompt{In}{incolor}{109}{\boxspacing}
\begin{Verbatim}[commandchars=\\\{\}]
\PY{c+c1}{// Gerando uma matriz simétrica}
\PY{n}{A4}\PY{+w}{ }\PY{p}{=}\PY{+w}{ }\PY{l+m+mi}{20}\PY{+w}{ }\PY{o}{*}\PY{+w}{ }\PY{n+nb}{rand}\PY{p}{(}\PY{l+m+mi}{3}\PY{p}{,}\PY{+w}{ }\PY{l+m+mi}{3}\PY{p}{)}\PY{+w}{ }\PY{o}{\PYZhy{}}\PY{+w}{ }\PY{l+m+mi}{10}\PY{p}{;}
\PY{n}{A4}\PY{+w}{ }\PY{p}{=}\PY{+w}{ }\PY{n+nb}{round}\PY{p}{((}\PY{n}{A4}\PY{+w}{ }\PY{o}{+}\PY{+w}{ }\PY{n}{A4}\PY{o}{\PYZsq{}}\PY{p}{)}\PY{o}{/}\PY{l+m+mi}{2}\PY{p}{)}
\PY{+w}{ }
\PY{c+c1}{// Vendo quais são seus autovalores}
\PY{n}{autovalores}\PY{+w}{ }\PY{p}{=}\PY{+w}{ }\PY{n+nb}{spec}\PY{p}{(}\PY{n}{A4}\PY{p}{)}

\PY{c+c1}{// Calculando esses autovalores com nossa função}
\PY{n}{S4}\PY{+w}{ }\PY{p}{=}\PY{+w}{ }\PY{n}{espectro}\PY{p}{(}\PY{n}{A4}\PY{p}{,}\PY{+w}{ }\PY{l+m+mi}{10}\PYZca{}\PY{p}{(}\PY{o}{\PYZhy{}}\PY{l+m+mi}{8}\PY{p}{))}
\end{Verbatim}
\end{tcolorbox}

    \begin{Verbatim}[commandchars=\\\{\}]
 A4  =
   5.  -5.  -1.
  -5.   4.   2.
  -1.   2.  -4.
 autovalores  =
  -4.4768213
  -0.3632067
   9.8400280
 S4  =
   9.8400280
  -4.4768213
  -0.3632067
    \end{Verbatim}

    \begin{tcolorbox}[breakable, size=fbox, boxrule=1pt, pad at break*=1mm,colback=cellbackground, colframe=cellborder]
\prompt{In}{incolor}{110}{\boxspacing}
\begin{Verbatim}[commandchars=\\\{\}]
\PY{n}{A5}\PY{+w}{ }\PY{p}{=}\PY{+w}{ }\PY{l+m+mi}{20}\PY{+w}{ }\PY{o}{*}\PY{+w}{ }\PY{n+nb}{rand}\PY{p}{(}\PY{l+m+mi}{4}\PY{p}{,}\PY{+w}{ }\PY{l+m+mi}{4}\PY{p}{)}\PY{+w}{ }\PY{o}{\PYZhy{}}\PY{+w}{ }\PY{l+m+mi}{10}\PY{p}{;}
\PY{n}{A5}\PY{+w}{ }\PY{p}{=}\PY{+w}{ }\PY{n+nb}{round}\PY{p}{((}\PY{n}{A5}\PY{+w}{ }\PY{o}{+}\PY{+w}{ }\PY{n}{A5}\PY{o}{\PYZsq{}}\PY{p}{)}\PY{o}{/}\PY{l+m+mi}{2}\PY{p}{)}
\PY{+w}{ }
\PY{n}{autovalores}\PY{+w}{ }\PY{p}{=}\PY{+w}{ }\PY{n+nb}{spec}\PY{p}{(}\PY{n}{A5}\PY{p}{)}

\PY{n}{S5}\PY{+w}{ }\PY{p}{=}\PY{+w}{ }\PY{n}{espectro}\PY{p}{(}\PY{n}{A5}\PY{p}{,}\PY{+w}{ }\PY{l+m+mi}{10}\PYZca{}\PY{p}{(}\PY{o}{\PYZhy{}}\PY{l+m+mi}{7}\PY{p}{))}
\end{Verbatim}
\end{tcolorbox}

    \begin{Verbatim}[commandchars=\\\{\}]
 A5  =
  -10.  -2.  -7.  -1.
  -2.    4.   1.   1.
  -7.    1.  -7.  -3.
  -1.    1.  -3.  -5.
 autovalores  =
  -16.316859
  -5.5016052
  -0.9370026
   4.7554671
 S5  =
  -16.316859
  -5.5016049
   4.7554669
  -0.9370026
    \end{Verbatim}

    \begin{tcolorbox}[breakable, size=fbox, boxrule=1pt, pad at break*=1mm,colback=cellbackground, colframe=cellborder]
\prompt{In}{incolor}{111}{\boxspacing}
\begin{Verbatim}[commandchars=\\\{\}]
\PY{n}{A6}\PY{+w}{ }\PY{p}{=}\PY{+w}{ }\PY{l+m+mi}{20}\PY{+w}{ }\PY{o}{*}\PY{+w}{ }\PY{n+nb}{rand}\PY{p}{(}\PY{l+m+mi}{5}\PY{p}{,}\PY{+w}{ }\PY{l+m+mi}{5}\PY{p}{)}\PY{+w}{ }\PY{o}{\PYZhy{}}\PY{+w}{ }\PY{l+m+mi}{10}\PY{p}{;}
\PY{n}{A6}\PY{+w}{ }\PY{p}{=}\PY{+w}{ }\PY{n+nb}{round}\PY{p}{((}\PY{n}{A6}\PY{+w}{ }\PY{o}{+}\PY{+w}{ }\PY{n}{A6}\PY{o}{\PYZsq{}}\PY{p}{)}\PY{o}{/}\PY{l+m+mi}{2}\PY{p}{)}
\PY{+w}{ }
\PY{n}{autovalores}\PY{+w}{ }\PY{p}{=}\PY{+w}{ }\PY{n+nb}{spec}\PY{p}{(}\PY{n}{A6}\PY{p}{)}

\PY{n}{S6}\PY{+w}{ }\PY{p}{=}\PY{+w}{ }\PY{n}{espectro}\PY{p}{(}\PY{n}{A6}\PY{p}{,}\PY{+w}{ }\PY{l+m+mi}{10}\PYZca{}\PY{p}{(}\PY{o}{\PYZhy{}}\PY{l+m+mi}{6}\PY{p}{))}
\end{Verbatim}
\end{tcolorbox}

    \begin{Verbatim}[commandchars=\\\{\}]
 A6  =
   9.   1.   5.   2.   5.
   1.  -9.   1.   4.  -1.
   5.   1.   5.   4.  -1.
   2.   4.   4.  -7.  -3.
   5.  -1.  -1.  -3.  -6.
 autovalores  =
  -12.599706
  -9.3512946
  -4.0616507
   4.2589108
   13.753740
 S6  =
   13.753740
  -12.599706
  -9.3512946
   4.2589013
  -4.0616413
    \end{Verbatim}

    Nossa função está funcionando perfeitamente!


    % Add a bibliography block to the postdoc
    
    
    
\end{document}
